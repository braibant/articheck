\documentclass[10pt]{beamer}
\usepackage[utf8]{inputenc}
\usepackage{courier}
\usepackage{txfonts}
\usepackage[english]{babel}
\usepackage{xcolor}
\usetheme{AnnArbor}
\usecolortheme{beaver}

\setbeamertemplate{headline}{}
% \setbeamertemplate{frametitle}{\insertframetitle}
\setbeamertemplate{navigation symbols}{}

\setbeamertemplate{itemize item}{\scriptsize\raise1.25pt\hbox{\donotcoloroutermaths$\blacktriangleright$}}
\setbeamertemplate{itemize subitem}{\tiny\raise1.5pt\hbox{\donotcoloroutermaths$\blacktriangleright$}}
\setbeamertemplate{itemize subsubitem}{\tiny\raise1.5pt\hbox{\donotcoloroutermaths$\blacktriangleright$}}
\setbeamertemplate{enumerate item}{\insertenumlabel.}
\setbeamertemplate{enumerate subitem}{\insertenumlabel.\insertsubenumlabel}
\setbeamertemplate{enumerate subsubitem}{\insertenumlabel.\insertsubenumlabel.\insertsubsubenumlabel}
\setbeamertemplate{enumerate mini template}{\insertenumlabel}

\title{Well-typed smart fuzzing}%
\author[T. Braibant]{Thomas Braibant \and Jonathan Protzenko \and Gabriel Scherer}
\institute[Cryptosense]{Cryptosense \qquad Inria}
\date[ML14]{ML Family Workshop}

\setbeamercovered{transparent}
%\setbeamerfont{frametitle}{size={\normalsize}}

% \usepackage[T1]{fontenc}
\usepackage{amsmath}
\usepackage{amssymb}
\usepackage{amsthm}
\usepackage{mathpartir}

\usepackage{listings}
\usepackage{graphicx}
\definecolor{ltblue}{rgb}{0,0.4,0.4}
\definecolor{dkblue}{rgb}{0,0.1,0.6}
\definecolor{dkgreen}{rgb}{0,0.4,0}
\definecolor{dkviolet}{rgb}{0.3,0,0.5}
\definecolor{dkred}{rgb}{0.5,0,0}
\usepackage{lstocaml}
\usepackage{tikz}
\usetikzlibrary{shapes,positioning,fit, shadows}

\begin{document}
% \newcommand \blue[1]{{\color{red!80!black}{#1}}}
\newcommand \orange[1]{{\color{orange}{#1}}}
% \newcommand \red[1]{{\color{red}{#1}}}
% \newcommand \grey[1]{{\color{gray}{#1}}}
% \newcommand \green[1]{{\color{violet}{#1}}}
% \newcommand \white[1]{{\color{white}{#1}}}

\newcommand\parenthesis[1] {
  \begin{flushright}
    {\scriptsize \em \color{dkred}{{{{ #1}}}}}
  \end{flushright}

}

\newcommand\remark[1] {
  \center{\bf #1}
}

\newcommand\plan[2]
{
\begin{frame}[plain]
  \begin{center}
    {\Huge  \sc #1} \\

    \vspace{1cm}

    #2
\end{center}
\end{frame}
}

\newcommand\fullpage[2]
{ % all template changes are local to this group.

    \begin{frame}[plain]
        \begin{tikzpicture}[remember picture, overlay]
          \node[at=(current page.center)]  {
                \includegraphics[width=\paperwidth, height=\paperheight, keepaspectratio]{#1}
            };
        \end{tikzpicture}
     \end{frame}
}

\tikzstyle{main} = [draw=dkred, very thick, rounded rectangle,inner sep=4pt, inner ysep=4pt, solid]

\begin{frame}
  \center
  \titlepage
\end{frame}

%%%  Intro
\fullpage{img/trading.jpg}{Test}
\fullpage{img/ncipher.jpg}{}     % Alexander Klink
\fullpage{img/luna.jpg}{}        % Alexander Klink

\begin{frame}[plain]
  \begin{columns}
    \column{0.45\textwidth}
    \includegraphics[width=\textwidth,keepaspectratio]{img/selfie.jpg}
    \center{\bf U.S. dept. of Justice}
    \column{0.45\textwidth}
    \center{\bf Two heists. }

    \center{\bf 5M\$ + 40M\$.}

    \center{\bf 5+12 accounts}

    \center{\bf 3+10 hours}
  \end{columns}
\end{frame}

\begin{frame}{Cryptosense workflow}
  \begin{center}
    \begin{tikzpicture}
      \draw (0,0) node[main,anchor=east] (n1) {Testing} (2,0)
      node[main,anchor=center] (n2) {Learning}; \draw[->] (n1) -- (n2)
      node[pos=0.5](mpd){};

      \node[main,anchor=center, right=1 of n2] (n3) {Model-checking};
      \draw[->] (n2) -- (n3);

      \node[main,anchor=center, right=1 of n3] (n4) {Monitoring};
      \draw[->] (n3) -- (n4);
    \end{tikzpicture}
  \end{center}
  \pause
  \remark{In this talk, I will focus on testing}
\end{frame}


\begin{frame}{Writing test cases is a pain.}

  \center{\bf Can I generate test cases automatically?}

  \begin{description}
  \item[\sc Yes:] QuickCheck.
    \begin{itemize}
    \item A combinator library to generate test cases in Haskell
      \parenthesis{and a host of other languages}
    \item Generate test cases for \alert{functions} based on their \alert{types}.
      \\
      E.g. \ocamle{val insert: int rbt -> int -> int rbt}
    \end{itemize}
  \item[\sc But:] ~\\
    \begin{itemize}
    \item What about testing the \alert{interaction} of several functions?
    \item How to deal with \alert{abstract} types?
    \end{itemize}
  \end{description}
\end{frame}

\begin{frame}{Articheck}

  \center{\bf A prototype library to generate test cases for safety properties}

  \medskip

  \begin{center}
    \begin{tabular}{|c|c|}
      \hline
      QuickCheck & Articheck \\
      \hline
      Whitebox (internal) & Blackbox (external) \\
      per function & at the API level \\
      generate random values & combine function calls \\
      use type definitions & use the interface\\
      \hline
    \end{tabular}
  \end{center}
\end{frame}

\begin{frame}{Leveraging APIs}

  \begin{itemize}
  \item Types help generate \alert{good} random values.
    \pause
  \item APIs help generate values that have the right \alert{internal invariants}.
    \pause
  \item Yet. Well-typed does not mean well-formed! We are still \alert{fuzzing}.
  \end{itemize}

\end{frame}

%%% Outline

\plan{Outline}{
  \begin{enumerate}
  \item Describing and testing APIs in a nutshell
  \item Taming the combinatorial explosion
  \item An industrial perspective: crypto APIs
  \end{enumerate}
}

%%% What's an API

\plan{Describing APIs}

\begin{frame}[fragile]
\frametitle{What's an API?}
  \begin{columns}
    \column{0.45\linewidth}
\begin{ocaml}
(* Mock Set API*)
type 'a t
val empty: 'a t
val insert: 'a -> 'a t -> 'a t
val delete: 'a -> 'a t -> 'a t
val mem: 'a -> 'a t -> bool
$ $
\end{ocaml}
    \column{0.5\linewidth}
\begin{ocaml}
(* Mock crypto API*)
type template
type key
type text = string
val generate_key: template -> key
val wrap_key: key -> key -> text
val encrypt: key -> text -> text
\end{ocaml}
\end{columns}

\remark{In this talk, APIs are first-order.}

\end{frame}

\begin{frame}[fragile]
  \frametitle{Testing an implementation of red-black trees}
  \begin{columns}
\column{0.45\linewidth}
\begin{ocaml}
(* Mock Set API*)
type 'a t
val empty: 'a t
val insert: 'a -> 'a t -> 'a t
val delete: 'a -> 'a t -> 'a t
val mem: 'a -> 'a t -> bool
$ $
\end{ocaml}
\column{0.45\linewidth}
\begin{tikzpicture}[%
  red/.style={circle, draw=none, rounded corners=1mm, fill=red, drop shadow,
        text centered, anchor=north, text=white, scale=0.5},
  black/.style={circle, draw=none, rounded corners=1mm, fill=black, drop shadow,
        text centered, anchor=north, text=white, scale=0.5},
  level distance=0.25cm,
  sibling distance=0.5cm,
  growth parent anchor=south
  ]
  \node [black] (1) {21}
    child {[sibling distance=0.25cm]node [red] {15}
      child {node [black] {11}}
      child {node [black] {17}}
    }
    child{[sibling distance=0.25cm]
      node [black] {51}
      child {
        node [red] {22}
      }
      child {
        node [red] {53}
      }
    };

    \node [black, right= 1cm of 1] (2) {3};

    \node [black, right= 1cm of 2] (3) {3}
    child{[sibling distance=0.25cm] node [red] {1}};

    \pause

    \node [black, right= 1cm of 3] (4) {3}
    child{[sibling distance=0.25cm] node [red] {1}}
    child{[sibling distance=0.25cm] node [red] {4}};
;
\end{tikzpicture}

\end{columns}
\end{frame}

\begin{frame}[fragile]
  \frametitle{A toy DSL for describing APIs}

\center{\bf Describing types.}
\begin{ocaml}
type 'a ty = {name: string; mutable content: 'a set; ...}
\end{ocaml}
\pause

\center{\bf Describing functions.}
\begin{ocaml}
type ('a,'b) fn =
| ret : 'a ty -> ('a,'a) fn
| @-> : 'a ty -> ('b,'c) fn -> ('a -> 'b,'c) fn
\end{ocaml}

\pause

\center{\bf Describing signatures.}
\begin{ocaml}
type elem = Elem: string * ('a,'b) fn * 'a -> elem
type signature = elem list
let declare label fn f = ...
\end{ocaml}
\end{frame}

\begin{frame}[fragile]
\frametitle{Coming back to red-black trees}

\begin{ocaml}
let int_ty : int ty = ...
let t_ty : (int RBT.t) ty  = ...
let api = [
  declare "empty"  (ret t_ty) RBT.empty;
  declare "insert" (int_ty @-> t_ty @-> ret t_ty) RBT.insert;
  declare "delete" (int_ty @-> t_ty @-> ret t_ty) RBT.delete;
  declare "mem"    (int_ty @-> t_ty @-> ret bool_ty) RBT.mem;
]
\end{ocaml}

\pause

\center{\bf Many possible strategies to combine these functions.}
\vspace{-1em}
\parenthesis{depth-first, breadth-first, round-robin, random walks}

{\bf This yields a list of inhabitants for the types:}
\begin{itemize}
\item used to check invariants (e.g., being a red-black tree)
\item used to check correctness properties
  \ocamle{$\forall$ x : t_ty, $\forall$ e : int_ty, delete x (insert x t) = t}
\end{itemize}

\end{frame}

\begin{frame}[fragile]
  \frametitle{A richer DSL for types}

\begin{ocaml}
type _ ty =
| Abstract: 'a abstract -> 'a ty
| Sum: 'a ty * 'b ty -> ('a,'b) Either.t ty
| Product: 'a ty * 'b ty -> ('a * 'b) ty
| Filter: 'a ty * ('a -> bool) -> 'a ty
| Bijection: 'a ty * ('a -> 'b) * ('b -> 'a) -> 'b ty
\end{ocaml}

\center{\bf Dolev-Yao style: A user/attacker can decompose sum and products}

\end{frame}
%%% DSL

%%% Combinatorial explosion

%%% Crypto APIs

\plan{Field report: Crypto APIs}{}

\begin{frame}{Some issues}

  \begin{itemize}
  \item We need to enumerate a big \alert{combinatorial space} made of:
    \begin{itemize}
    \item constants (types whose inhabitants are known, e.g. templates, key types, mechanisms)
    \item variables (types populated dynamically, e.g. handles, ciphertexts)
    \end{itemize}
  \item We want a good coverage (``smaller values'' first)
  \item We want a reproducible behavior (i.e., avoid random testing)
  \end{itemize}
\end{frame}

\begin{frame}{Feature \#1: a library of enumerators}
  \begin{description}
  \item[\sc Low memory footprint:] if possible, constant space
  \item[\sc Efficient access:] if possible, constant time for random accesses
  \item[\sc Some bad solutions:] ~
    \begin{itemize}
    \item Arrays: one cell per element, contiguous, ...
    \item Lists: one cell per element, linear time accesses, ...
    \item Lazy lists: idem, ...
    \end{itemize}
\end{description}
\end{frame}

\begin{frame}[fragile]
  \frametitle{Quick peek}

\begin{ocaml}
type 'a t = {size: int; nth: int -> 'a}

val of_list     : 'a list -> 'a t
val product     : 'a t -> 'a t -> ('a * 'b) t
val map         : ('a -> 'b) -> 'a t -> 'b t
val append      : 'a t -> 'a t -> 'a t
...
val subset      : 'a t list -> 'a list t t
...
val squash      : 'a t t -> 'a t
val round_robin : 'a t t -> 'a t
\end{ocaml}

\ocamle{subset} is paramount to generate templates (list of
attributes) efficiently.

\end{frame}

\begin{frame}[fragile]
  \frametitle{Feature \#2: a library for combinatorial enumeration}

\begin{ocaml}
(* Combinators to describe the combinatorial problem *)
type 'a d
val constant: 'a Enum.t -> 'a d
val variable: 'a set    -> 'a d
val filter  : ('a -> bool) -> 'a d -> 'a d
val map     : ('a -> 'b) -> 'a d -> 'b d
val sum     : 'a d -> 'b d -> ('a,'b) Either.t d
val product : 'a d -> 'b d -> ('a * 'b) d
\end{ocaml}
\pause
\begin{ocaml}
(* Imperative updates to the state of the exploration*)
type 'a t
val start: 'a d -> 'a t
val take : 'a t -> 'a option
val add  : 'a -> 'a set -> unit
\end{ocaml}

\begin{itemize}
\item The state is roughly an enumerator and an index.
\item When empty, compute an enumerator for the next elements to process
  \begin{itemize}
  \item avoiding redundancy with what has been done already;
  \item maximizing \alert{throughput}
  \end{itemize}
\end{itemize}
\end{frame}
\begin{frame}[fragile]
  \frametitle{Quick peek}

  Compute the effect of adding a new element $v$ to a variable $\alpha$.
  \begin{displaymath}
  \begin{array}{rcl}
    \delta_\alpha(k) & \triangleq & \emptyset \\
    \delta_\alpha(\alpha) & \triangleq & \left\{v\right\} \\
    \delta_\alpha(\beta)  & \triangleq & \emptyset \\
    \delta_\alpha(f~b)    & \triangleq & f (\delta_\alpha{b}) \\
    \delta_\alpha(a \times b)    & \triangleq &
    \delta_\alpha(b) \times c  \cup b \times \delta_\alpha(c) \\
  \end{array}
\end{displaymath}
\end{frame}

\begin{frame}[fragile]
  \frametitle{Testing an HSM}
\begin{scriptsize}
\begin{tabular}{|l|r|r|r|r|r|}
\hline
\verb!C_GetTokenInfo!                  &      1 &      1 &        &      0.001 &         0\\
\verb!C_GenerateKey!                   &  30095 &  10705 &  19390 &      1.730 &         0\\
\verb!C_GenerateKeyPair!               & 116563 &  22478 &  94085 &    100.224 &   9382739\\
\verb!C_CreateObject (DES)!            &   6019 &   2141 &   3878 &      0.339 &         0\\
\verb!C_CreateObject (DES3)!           &   6019 &   2141 &   3878 &      0.357 &         0\\
\verb!C_CreateObject (AES)!            &  18057 &   6423 &  11634 &      1.236 &         0\\
\verb!C_CreateObject (RSA, public)!    &   1161 &    577 &    584 &      0.143 &         0\\
\verb!C_CreateObject (RSA, private)!   &   4091 &   1199 &   2892 &      1.122 &         0\\
\verb!C_Encrypt (with symmetric key)!  & 110160 &   4716 & 105444 &      5.249 &         0\\
\verb!C_Encrypt (with asymmetric key)! &  67648 &  37730 &  29918 &      6.691 &        47\\
\verb!C_Decrypt (with symmetric key)!  & 116466 &    111 & 116355 &      9.424 &    360498\\
\verb!C_Decrypt (with asymmetric key)! & 116563 &     82 & 116477 &    149.626 &    250910\\
\verb!C_WrapKey (with symmetric key)!  & 116563 &   1824 & 114737 &      9.833 &   8198753\\
\verb!C_WrapKey (with asymmetric key)! & 116563 &  72862 &  43674 &    119.661 & 368429621\\
\verb!C_GetAttributeValue (key value)! &  57469 &  24736 &  32733 &     19.800 &         0\\
\verb!C_SetAttributeValue!             & 116562 &   3909 & 112653 &     17.796 & 387844143\\
\hline
\end{tabular}
\end{scriptsize}

\center{\bf $10^6$ tests in 540s. 20\%overhead. 0 failures.}


\end{frame}

\begin{frame}{Take away}

  \remark{We have a principled way to test (persistent) APIs.}

\end{frame}

\plan{One more thing about Cryptosense}{
  We find bugs in HSMs. \\
  We find logical attacks on crypto APIs. \\
  We are recruiting.
}

\end{document}
