\documentclass[twocolumn,9pt]{article}
\usepackage{titlesec}

\titleformat*{\section}{\large\bfseries}
\titleformat*{\subsection}{\bfseries}

\usepackage{hyperref}
\usepackage[T1]{fontenc}
\usepackage[utf8x]{inputenc}
\usepackage{minted}
\usepackage{xspace}
% \usepackage{microtype}
\usepackage{paralist}
\usepackage{lmodern}

\newcommand{\acheck}{ArtiCheck\xspace}
\newcommand{\qcheck}{QuickCheck\xspace}
\newcommand{\csgen}{Generator\xspace}

\newcommand{\code}[1]{\texttt{#1}}
\newcommand{\sref}[1]{\S\ref{#1}}
\newminted{ocaml}{fontsize=\footnotesize}
\renewcommand\paragraph[1]{\newline\textbf{#1}}
\begin{document}
\date{}

\title{\vspace{-2cm}Well-typed generic fuzzing for APIs}
\author{Thomas Braibant$^{1,2}$
  \quad Jonathan Protzenko$^{2}$
  \quad Gabriel Scherer$^{2}$ \\
  ${}^1$ : Cryptosense \quad ${}^2$ : Inria
}

\maketitle


\section{Introduction}
Despite recent advances in program certification, testing remains a
widely-used component of the software development cycle. Various
flavors of testing exist: popular ones include \emph{unit testing},
which consists in manually crafting test cases for specific parts of
the code base, as well as \emph{QuickCheck-style} testing, where
instances of a type are automatically generated to serve as test
inputs.
%
These methods of testing can be thought of as \emph{internal} testing:
the test routines need to access the internal representation of the
data-structures that are used by the functions under test. They can
also be thought of as \emph{per-function} testing: a test suite is
built (by hand, or automatically) for each function that must be
tested.

We propose a new method of \emph{external} testing that applies at the
level of the \emph{module interface}. The core of our work is a small
embedded domain specific language to describe APIs, i.e., functions
and data-types. Then, these API descriptions are used to drive the
generation of test-cases.
%
We have successfully used this method in two different contexts:
%
\\
\textbf{Test case generation.} First, we implemented a library dubbed
\acheck that combines the functions exported by a given module
interface to build elements of the various data-types exported by the
module, and then checks that all the elements of these data-types meet
user-defined invariants.
\\
\textbf{Smart fuzzing.} Second, the first author re-implemented this
methodology while working at Cryptosense to automate the analysis of
(security) APIs. More precisely, Cryptosense's library uses an API
description to automatically exercise vendors' implementations of the
said API.

\section{The essence of external testing}\label{sec:example}
In the present section, we illustrate the essential idea of external
testing of APIs through a simple example in the context of OCaml. We
consider a module \code{SIList} whose type \code{t} represents sorted
lists of integers. This invariant is maintained by making \code{t}
abstract: the user is forced to use functions from the interface to
build instances of the type \code{t}.
\begin{ocamlcode}
module SIList: sig
  type t
  val empty: t
  val add: t -> int -> t
  val sorted: t -> bool
end
\end{ocamlcode}
Given this signature, all a client can do is combine calls to
\code{empty} and \code{add} to build new instances of \code{t}. We
take the point of view that a client should not be able to violate
type abstractions, e.g., building ill-formed applications like
\code{add empty empty}. (This is obvious in the context of OCaml, but
we also apply this rule to the security APIs we consider.)
%
In essence, we want to represent well-typed applications of function
symbols such as \code{empty} and \code{add}, and thus, we need to
reify the types of these two functions as an OCaml data-type. This can
be done using generalised abstract data-types (GADTs). For instance,
we can define the type \code{('a,'b) fn} of \emph{function
  signatures}.  An element \code{('a,'b) fn} describes a function that
has type \code{'a} and generates values of type \code{'b}.
\begin{ocamlcode}
type (_,_) fn =
| Ret: 'a ty -> ('a, 'a) fn
| Fun: 'a ty *  ('b, 'c) fn -> ('a -> 'b, 'c) fn
and 'a ty = {ident: string; mutable enum: 'a list}
let (@->) a b = Fun (a,b)
let ret a = Ret a
\end{ocamlcode}
The type \code{'a ty} is used to keep track of the instances of
\code{'a} that have been built. For types like \code{int}, we can
populate the enumeration before-hand. For the abstract types like
\code{t}, we can only populate this enumeration on the fly. Now, we
can simply describe an API as an heterogeneous list of functions:
\begin{ocamlcode}
type signature = elem list
and  elem = Elem : string * ('a,'b) fn * 'a -> elem
let declare label fn f = Elem (label,fn,f)
\end{ocamlcode}
The API of the \code{SIList} module can be encoded as follows.
\begin{ocamlcode}
let int_ty : int ty = ...
let t_ty : SIList.t ty = ...
let api =
  let open SIList in
  [ declare "empty" (ret t_ty) empty;
    declare "add" (int_ty @-> t_ty @-> ret t_ty) add]
\end{ocamlcode}
Then, we exercise the API, applying the various functions it exports
to suitable arguments. In effect, this may produce new suitable
arguments that we can use to continue the test process; or, we may
reach a fixpoint when the set of instances of types that we have
cannot be used to produce new instances. (Of course, we can also
ensure termination by putting a bound on the number of instances we
want to produce.)  What is important is that the process of exercising
the functions of an API is closely related to the process of building
instances of its types. In what follows, \emph{testing} will denote
one or the other.

\section{Implementing a testing library}
The simplistic design that is outlined in \sref{sec:example} has many
shortcomings. Here are some hints about the improvements we
implemented.
%
\paragraph{A better algebra of types.} If the type of a function
indicates that it returns a pair, we want to be able to break it
apart, and use its components separately. The same goes for sums, we
want to be able to discriminate between the two possible head
constructors. Thus, we enrich the definition of \code{'a ty} to
support sum, products, and atoms (arbitrary user-defined types that
are used in an abstract manner). We also give ways to encode n-ary
sums and records via two-ways mappings (e.g., we encode \code{'a
  option} as \code{() + 'a}).
%
\paragraph{Storing instances.} Using lists to store instances of
atomic types is inefficient, but there is no clear ``one size fits
all'' solution. The right way to store instances of these types
actually depends on the use-case.
%
Sometimes, the user needs a set of the instances that may be
quotiented by a suitable equivalence relation.
%
Sometimes, keeping a sample of the instances that are created is
sufficient.
%
Sometimes, the user is only interested in enumerating a known set of
values to bootstrap the creation of other instances.
%
We propose various kinds of \emph{containers} to store instances, and
we make it possible for the user to choose the right one depending on
the situation.
%
\paragraph{Iterating tests.} The problem of testing functions over all
possible inputs can nicely be presented as a kind of fixpoint
computation. If we define the state of the iteration as a mapping from
types to the set of their inhabitants, then an API can be considered
as a simple state transformer.
%
This presentation makes it possible to use an off-the-shelf fixpoint
library, F. Pottier's \code{Fix} in \acheck as a first
approximation.
%
However, using \code{Fix} does not scale up to the use-cases of
Cryptosense's library. We solve this issue in two steps. First, we
implement a library of \emph{lazy enumerators} to store the set of
test cases that need to be processed for each function. Then, we use
an algorithm akin to formal derivatives to compute what is the extra
work that should be done each time a value is added to a set of
instances. Lazy enumerators and derivatives are crucial to mitigate
the combinatorial explosion.

\section{Use-cases}
\subsection{Examples with \acheck}
We used \acheck on various examples: Red-Black trees, AVL trees,
skew-heaps and BDDs. For instance, we checked that all the BDDs
created by our library are reduced and ordered.
\subsection{Examples at Cryptosense}
We are using Cryptosense's library to reverse engineer (i.e., test)
the behaviour of hardware devices that implement security APIs.

\section{Conclusion}
This talk will expose the key abstractions that are used in the design
of our two libraries and highlight the lessons learned.


% \bibliographystyle{plain}
% \hbadness=10000
% \bibliography{local}

\end{document}
