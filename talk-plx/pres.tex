\documentclass[final,xetex]{beamer}

\usepackage{fontspec}
\usepackage{xunicode}
  \setsansfont{Museo Sans 500}
  \setromanfont{Museo}
  \setmonofont[Scale=.88]{DejaVu Sans Mono}

\input{tango-colors.tex}

% ------------------------------------------------------------------------------

\usepackage{minted}
  \usemintedstyle{tango}
\usepackage{xspace}
\usepackage{ucharclasses}
  \setTransitionsFor{Dingbats}{\begingroup\fontspec{DejaVu Sans}[Scale=MatchLowercase]}{\endgroup}

\usepackage{tikz}
  \usetikzlibrary{tikzmark}
  \usetikzlibrary{calc}
  \usetikzlibrary{shapes,arrows}
  \usetikzlibrary{fit}          % fitting shapes to coordinates
  \usetikzlibrary{positioning}
  \usetikzlibrary{backgrounds}  % drawing the background after the foreground

%\def\mintedusecache{}
\def\mintedwritecache{}

\newcommand{\MzITitle}[1]{
  \begin{tikzpicture}
    \node [
      draw,
      thin,
      aluminium5,
      fill=aluminium1,
      shape=rectangle,
      opacity=.9,
      overlay,
      text width=7cm,
      minimum height=3em,
      font=\Large,
      text centered
    ] {
      {\color{black}#1}
    };
  \end{tikzpicture}
}

\mode<presentation> {
  \setbeamercolor*{Title bar}{fg=white,bg=skyblue2}
  \setbeamercolor*{frametitle}{parent=Title bar}
  \setbeamercolor*{block title}{bg=skyblue2,fg=white}
  \setbeamercolor*{block body}{bg=aluminium1,fg=aluminium5}
  \setbeamercolor*{normal text}{bg=white,fg=aluminium5}
  \setbeamercolor*{palette tertiary}{fg=white,bg=chameleon3}
  \setbeamercolor*{palette secondary}{bg=aluminium6,fg=white}
  \setbeamercolor*{palette primary}{fg=aluminium6,bg=aluminium1}
}
\usecolortheme[named=chameleon3]{structure}
\mode<presentation> {
  \setbeamertemplate{navigation symbols}{}
  \setbeamertemplate{items}[circle]
  \setbeamercolor*{Title bar}{bg=white,fg=chameleon3}
  \setbeamersize{text margin left=2em}
  \setbeamertemplate{footline}{
    \leavevmode%
    \hbox{%
    \begin{beamercolorbox}[wd=.3\paperwidth,ht=2.25ex,dp=1ex,center]{author in head/foot}%
      \usebeamerfont{author in head/foot}\insertshortauthor\ —\ \insertshortinstitute
    \end{beamercolorbox}%
    \begin{beamercolorbox}[wd=.4\paperwidth,ht=2.25ex,dp=1ex,center]{title in head/foot}%
      \usebeamerfont{title in head/foot}\insertshorttitle
    \end{beamercolorbox}%
    \begin{beamercolorbox}[wd=.3\paperwidth,ht=2.25ex,dp=1ex,right]{date in head/foot}%
      \usebeamerfont{date in head/foot}\insertshortdate{}\hspace*{2em}
      \insertframenumber{} / \inserttotalframenumber\hspace*{2ex}
    \end{beamercolorbox}}%
    \vskip0pt%
  }
}

\newcommand{\ac}{\textsf{ArtiCheck}\xspace}
\newcommand{\red}[1]{{\color{orange2}#1}}
\newcommand{\green}[1]{{\color{chameleon3}#1}}
\newcommand{\code}[1]{\texttt{\textbf{#1}}}
\newcommand{\newauthor}[2]{
  \parbox{0.4\textwidth}{
    \texorpdfstring
      {
        \centering
        #1 \\
        {\scriptsize{\urlstyle{same}\url{#2}\urlstyle{tt}}}
      }
      {#1}
  }
}
\newminted{ocaml}{fontsize=\footnotesize,escapeinside=!!}

\title[\ac]{
\ac: well-typed fuzzing of module APIs
}
\author[Jonathan Protzenko]{
  \newauthor{Jonathan Protzenko}{protz@microsoft.com}
}

\institute{PL(X)}
\date{March 18\textsuperscript{th}, 2015}

\begin{document}

\begin{frame}
  \begin{center}
  {\fontsize{50}{50}\selectfont\hspace{-1.7em}\ac}
  \end{center}

  \vspace{2cm}
    Thomas Braibant\\
    Jonathan Protzenko\\
    Gabriel Scherer
\end{frame}

\begin{frame}{What is testing?}
  \includegraphics[width=.8\linewidth]{dawg.jpg}

  \vspace{2em}

  Let's make it a little bit smarter than that.
\end{frame}

\begin{frame}[plain]
  \begin{center}\MzITitle{The basics of the library}\end{center}
\end{frame}

\begin{frame}[fragile]{Running example}
  \begin{ocamlcode}
(* tree.mli *)
type t
val empty: t
val add: t -> int -> t
val remove: t -> int -> t

val check: t -> bool
  \end{ocamlcode}

\bigskip

\pause
We want to act \emph{as a fake user} of the library.

\bigskip

\pause
\emph{External} testing \emph{vs.} \emph{internal} testing
\end{frame}

\begin{frame}{Good call \emph{vs.} bad call}
  Only \emph{certain} calls are \emph{well-typed}.

  \begin{itemize}
    \item \code{add empty 1} = \green{GOOD}
    \item \code{add add add} = \red{BAD}
  \end{itemize}
\end{frame}

\begin{frame}[fragile]{Getting type-theoretic (1)}
  GADTs! Describing well-typed calls.

\bigskip

\begin{ocamlcode}
  type (_, _) fn =
  | Ret: 'a ty -> ('a, 'a) fn
  | Fun: 'a ty * ('b , 'c) fn -> ('a -> 'b, 'c) fn
\end{ocamlcode}

\bigskip

The type \code{('a, 'b) fn} describes a function with arrow type \code{'a},
whose return type is \code{'b}.

\end{frame}

\begin{frame}[fragile]{Getting type-theoretic (2)}
  Type descriptors.

  \bigskip

  \begin{ocamlcode}
type 'a ty = {
  mutable enum: 'a list;
  fresh: ('a list -> 'a) option;
}
  \end{ocamlcode}

  \bigskip

  The type \code{'a ty} describes a \emph{collection of instances} for type
  \code{'a}.

  \bigskip

  \begin{itemize}
    \item For \code{int}: \code{fresh} generates a fresh integer each time.
    \item For \code{t}: no \code{fresh} function.
  \end{itemize}
\end{frame}

\begin{frame}[fragile]{Evaluating!}
\begin{ocamlcode}
let rec eval : type a b. (a,b) fn -> a -> b list =
  fun fd f ->
    match fd with
    | Ret _ -> [f]
    | Fun (ty,fd) -> List.flatten (
        List.map (fun e -> eval fd (f e)) ty.enum)
let rec codom : type a b. (a,b) fn -> b ty =
  function
    | Ret ty -> ty
    | Fun (_,fd) -> codom fd
\end{ocamlcode}
\end{frame}

\begin{frame}[fragile]{Registering new instances}
\begin{ocamlcode}
let use (fd: ('a, 'b) fn) (f: 'a): unit =
  let prod, ty = eval fd f, codom fd in
  List.iter (fun x ->
    if not (List.mem x ty.enum)
    then ty.enum <- x::ty.enum
  ) prod
\end{ocamlcode}

\end{frame}

\begin{frame}[fragile]{Declaring an interface}
\begin{ocamlcode}
type sig_elem = Elem : ('a,'b) fn * 'a -> sig_elem
type sig_descr = (string * sig_elem) list

let tree_t : Tree.t ty  = ...
let int_t = ... (* integers use a [fresh] function*)

let sig_of_tree = [
  ("empty", Elem (returning tree_t, Tree.empty));
  ("add", Elem (tree_t @-> int_t @-> returning tree_t, Tree.add)); ]
let _ =
  Arti.generate sig_of_tree;
  assert (Arti.counter_example tree_t Tree.check = None)
\end{ocamlcode}

\end{frame}

\begin{frame}[plain]
  \begin{center}\MzITitle{Where the trouble begins}\end{center}
\end{frame}

\end{document}
