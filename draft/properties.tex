\section{Expressing correctness properties}
\label{sec:properties}

We mentioned in the \code{SIList} example the \code{counter\_example} function.
% TODO: remove the signature?
%
\begin{ocamlcode}
val counter_example: 'a pos -> ('a -> bool) -> 'a option
\end{ocamlcode}
%
The function takes a description of some (positive) datatype
\code{'a}, iterates on the generated instances of this type and checks
that a predicate \code{'a -> bool} holds for all instances, or returns
a counter-example otherwise. At a more abstract level, this means that we are
checking a property of the form \[ \forall (x \in t), T(x) \] where
$T(x)$ is simply a boolean expression.
Multiple quantifiers can be simulated through the use of product types, such as
in the typical formula of association maps:
%
\[\begin{array}{l}
  \forall (m \in \mathtt{map}(K,V))\ \forall (k \in K)\ \forall (v \in V),\\
  \qquad \mathtt{lookup(k,insert(k,v,m)) = v}
\end{array}\]
%
which can be expressed as follows (where \code{*@} is the operator for creating
product type descriptors):
%
\begin{ocamlcode}
  let lookup_insert_prop (k, v, m) =
    lookup k (insert k v m) = v
  let () = assert (None =
    let kvm_t = k_t *@ v_t *@ map_t in
    counter_example kvm_t lookup_insert_prop)
\end{ocamlcode}

One then naturally wonders what a good language would be for describing the
correctness properties we wish to check. In the example above, we naturally
veered towards first-order logic, so as to express formulas with only prenex,
universal quantification. The universal quantifiers are to be understood with a
``test semantics'', that is, they mean to quantify over all the random instances
we generated.
%
Can we do better? In particular, can we capture the full language of first-order
logic, as a reasonable test \emph{description language} for a practical
framework?

It feels natural to use first-order logic as a specification language
in the context of structured verification, such as with SMT solvers or
a finite model finder~\cite{DBLP:conf/itp/BlanchetteN10}.  However,
supporting full first-order logic as a specification language for
randomly-generated tests is hard for various reasons.

For instance, giving ``test semantics'' to an existentially-quantified formula
such as $\exists(x \in t). T(x)$ is awkward. Intuitively, there is not much
meaning to the formula. The number of generated instances is finite; that none
satisfies $T$ may not indicate a bug, but rather that the wrong elements have
been tested for the property. Conversely, finding a counter-example to a
universally-quantified formula \emph{always} means that a bug has been found.
%
Trying to distinguish absolute (positive or negative) results from probabilistic
results opens a world of complexity that we chose not to explore.

Surprisingly enough, there does not seem to be a consensus in the
literature about random testing for an expressive, well-defined subset of
first-order logic. The simplest subset one directly thinks of is
formulas of the form:
\[ \forall x_1 \dots x_n, P(x_1, \dots, x_n) \Rightarrow T(x_1, \dots,
x_n) \] where $P(x_1, \dots, x_n)$ (the \emph{precondition}) and
$T(x_1, \dots, x_n)$ (the \emph{test}) are both quantifier-free
formulas.

The reason this implication is given a specific status is to make it possible to
distinguish tests that succeeded because the \emph{test} was effectively
successful from tests that succeeded because the \emph{precondition} was not
met. The latter are ``aborted'' tests that bring no significant value, and
should thus be disregarded. In \arti, we chose to restrict ourselves to this
last form of formulas.
