\section*{Conclusion}

We have presented the design of \arti, a novel library that allows one to check
the invariants of a module signature by simulating user interaction with the
module. \arti behaves like a fake client: it calls functions, constructs and
destructs products or sums, and for each element check that the invariants are
verified. The key to performing this in a generic, abstract manner relies on
GADTs, which abstract the different types that may be manipulated into a common
representation.

We identified various performance problems that arise. The library handles them
via a symbolic representation of types in combination with a little bit of
mutable state to avoid handling large, intermediary results in memory.

The result is a self-contained library that wraps the core concepts of
\emph{external testing} and offers clients a cheap and efficient way to test
their programs. The library, for instance, successfully detects infamous issues
such as the AVL re-balancing issue in the standard library of OCaml, with a much
lower cost than a complete machine-assisted verification of the module.

While the library exposes the essence of \emph{external testing} and has already
proven worthwhile, we believe there is potential for improvement and expansion
into a fully-fledged testing library. The code, along with the entire history of
the present paper, are available online at
\url{https://github.com/braibant/articheck}.
