\documentclass[nonatbib]{sigplanconf}

\usepackage[T1]{fontenc}
\usepackage[utf8x]{inputenc}
\usepackage{minted}
\usepackage{xspace}


\newcommand{\arti}{\textsf{ArtiCheck}\xspace}
\newcommand{\qcheck}{QuickCheck\xspace}
\newcommand{\code}[1]{\texttt{#1}}
\newcommand{\sref}[1]{\S\ref{#1}}
\newminted{ocaml}{}

\begin{document}

% For ACM, explicitly specify A4 paper.
% For us, no need to specify it (but check your TeX configuration if
% you insist on producing a specific size).
\special{papersize=8.5in,11in}
\setlength{\pdfpageheight}{\paperheight}
\setlength{\pdfpagewidth}{\paperwidth}

\exclusivelicense

\conferenceinfo{ICFP~'14}{September 1--3, 2014, Copenhagen, Denmark}
\copyrightyear{2014}
\copyrightdata{978-1-nnnn-nnnn-n/yy/mm}
\doi{nnnnnnn.nnnnnnn}

\title{Functional pearl: zero-knowledge testing for module interfaces}

\authorinfo{
  Thomas Braibant \quad
  Jacques-Henri Jourdan \quad
  Jonathan Protzenko \quad
  Gabriel Scherer
}{
  INRIA
}{http://gallium.inria.fr/blog/}
% \authorinfo{Thomas Braibant}
%            {INRIA}
%            {thomas.braibant@inria.fr}
% \authorinfo{Jacques-Henri Jourdan}
%            {INRIA}
%            {jacques-henri.jourdan@inria.fr}
% \authorinfo{Jonathan Protzenko}
%            {INRIA}
%            {jonathan.protzenko@ens-lyon.org}
% \authorinfo{Gabriel Scherer}
%            {INRIA}
%            {gabriel.scherer@inria.fr}

\maketitle

\begin{abstract}
  In spite of recent advances in full program certification, testing remains a
  widely-used component of the software development cycle. Various flavors of
  testing exist: popular ones include \emph{unit testing}, which consists in
  manually crafting test cases for specific parts of the code base, as well as
  \emph{quickcheck-style} testing, where instances of a type are automatically
  generated to serve as test inputs.

  These classical methods of testing can be thought of as \emph{internal}
  testing: the test routines access the internal representation of whatever
  module should be checked. We propose a new method of \emph{external} testing
  where test code checks an \emph{abstract} data structure. Our new testing
  method takes a description of a \emph{module signature}, then builds sequences
  of function calls that generate elements of the abstract type just like any
  other client code. Counter-examples, if any, are then presented to the user.
\end{abstract}

\category{CR-number}{subcategory}{third-level}

\keywords{
functional programming,
testing,
quickcheck
}

\section{Introduction}
\label{sec:introduction}

Software development is hard. Industry practices still rely, for the better
part, on tests to ensure the functional correctness of programs. Even in more
sophisticated circles, such as the programming language research community, not
everyone has switched to writing all their programs in Coq. Testing is thus a
cornerstone of the development cycle. Moreover, even if the end goal is to fully
certify a program using a proof assistant, it is still worthwhile to eliminate
bugs early by running a cheap, efficient test framework.

Testing boils down to two different processes: generating test cases
for test suites; then verifying that user-written assertions and
specifications of program parts are not falsified by the test suites.

\qcheck{}~\cite{DBLP:conf/icfp/ClaessenH00} is a popular, efficient
tool for that purpose. First, it provides a combinator library based
on type-classes to build test case generators. Second, it provides
a principled way for users to specify properties over
functions. For instance, users may write predicates such as ``reverse
is an involution''. Then, the \qcheck framework is able to create
\emph{instances} of the type being tested, e.g. lists of integers.
The predicate is tested against these test cases, and any
counter-example is reported to the user.

Our novel approach is motivated by some limitations of the \qcheck
framework.
%
First, data-structures often have \emph{internal invariants} that
would not be broken when using the API of the data-structure. Thus,
testing one particular function of such an API requires generating
test-cases that meet these invariants.
%
Yet, writing \emph{good} generators for (involved) data-structures is
tedious, or even plain hard. Consider the simple problem of generating
binary search trees (BST), for instance. Being smarter than merely
generating trees and filtering out those which are not
search trees requires reimplementing a series of insertions and
deletions into BSTs. But these functions are most certainly already
part of the code that is tested!
%

We argue that this low-level manipulation could be taken care of by
the testing framework. That is, we argue that the framework should be
able, by itself, to combine functions exported by the module we wish
to test in order to build instances of the data-types defined in the
same module. If the module exports the properties and the invariants
that should hold, then the testing framework needs not see the
implementation.
%
In a nutshell, we want to move towards an \emph{external testing} of abstract
modules.

In the present document, we describe a library that does precisely
that, dubbed \arti. The library is written in OCaml. While \qcheck
uses type-classes as a host language feature to conduct value
generation, we show how to implement the proof search in library
code -- while remaining both type-safe and generic over the
tested interfaces, thanks to GADTs.

\section{The essence of external testing}
\label{sec:essence}

In the present section, we illustrate the essential idea of external testing
through a simple example, which is that of a module \code{SIList} whose type
\code{t} represents sorted integer lists. The invariant is maintained by making
\code{t} abstract and requiring the user to go through the exported functions
\code{empty} and \code{add}.

This section, unfolding from the initial example, introduces the key ideas of
external testing: a GADT type that describes well-typed applications in the
simply-typed lambda calculus; a description of module signatures that we wish to
test; type descriptors that record all the instances of a type that we managed
to construct.

The point of view adopted in this section is intendedly simplistic. The design,
as presented here, contains several obvious shortcomings. It allows,
nonetheless, for a high-level overview of our principles, and paves the way for
a more thorough discussion of our design (\sref{sec:representation}).

Here is the signature for our module of sorted integer lists.
The \code{check} function represents the \emph{invariant} that the module
intends to preserve. The module admits a straightforward implementation, which
we also show.
%
% 7dd765501e4d12d98e9d90a0ece490486cfe6913
\begin{ocamlcode}
module type SIList: sig
  type t
  val empty: t
  val add: t -> int -> t
  val check: t -> bool
end = struct
  type t = int list
  let empty = []
  let rec add x = function
    | [] -> [x]
    | t::q -> if t<x then t::add x q else x::t::q
  let rec check = function
    | [] | [_] -> true
    | t1::(t2::_ as q) -> t1 <= t2 && check q
end
\end{ocamlcode}
%
Roughly speaking, our goal is to generate, as if we were \emph{client code} of
the module, instances of type \code{t} using only the functions exported by the
module. Therefore, one of our first requirements is a data structure for keeping
track of the \code{t}'s created so far. We also need to keep track of the
integers we have generated so far, since they are necessary to call the
\code{add} function: \arti will thus manipulate several \code{ty}'s for all the
types it handles.
%
\begin{ocamlcode}
type 'a ty = { (* Other implementation details omitted *)
  mutable enum: 'a list;
  fresh: ('a list -> 'a) option; }
\end{ocamlcode}
%
A type descriptor \code{'a ty} keeps track of all the \emph{instances} of
\code{'a} we have created so far in its \code{enum} field. Built-in types such
as \code{int} do not belong to the set of types whose invariants we wish to
check. For such types, we provide a \code{fresh} function that generates an
instance different from all that we have generated so far.

From the point of view of the client code, all we can do is combine
\code{add} and \code{empty} to generate new instances. \arti, as a fake client,
should thus behave similarly and automatically perform repeated applications of
\code{add} so as to generate new instances. We thus need a description of what
combinations of functions are legal for \arti to perform.

In essence, we want to represent well-typed applications in the
simply-typed lambda-calculus. This can be embedded in OCaml using generalized
algebraic data types (GADTs). We define the GADT \code{('a, 'b) fn}, which
describes ways to generate instances of type \code{'b} using a function of type
\code{'a}. We call it a \emph{function descriptor}.
%
\begin{ocamlcode}
type (_,_) fn =
| Ret: 'a ty -> ('a,'a) fn
| Fun: 'a ty * ('b, 'c) fn -> ('a -> 'b, 'c) fn
(* Helpers for creating [fn]'s. *)
let (@->) ty fd = Fun (ty,fd)
let returning ty = Ret ty
\end{ocamlcode}
%
The \code{Ret} case describes a constant value, which has type \code{'a} and
produce one instance of type \code{'a}. For reasons that will soon become
apparent, we also record the descriptor of type \code{'a}. \code{Fun} describes
the case of a function from \code{'a} to \code{'b}: using the descriptor of type
\code{'a}, we can apply the function to obtain instances of type \code{'b};
combining that with the other \code{('b, 'c) fn} gives us a way to produce
elements of type \code{'c}, hence the \code{('a -> 'b, 'c) fn} conclusion.
%
\begin{ocamlcode}
let rec eval : type a b. (a,b) fn -> a -> b list =
  fun fd f ->
    match fd with
    | Ret _ -> [f]
    | Fun (ty,fd) -> List.flatten (
        List.map (fun e -> eval fd (f e)) ty.enum)
let rec codom : type a b. (a,b) fn -> b ty =
  function
    | Ret ty -> ty
    | Fun (_,fd) -> codom fd
\end{ocamlcode}
%
The \code{eval} function is central: taking a function descriptor \code{fd}, it
recurses over it, thus refining the type of its argument \code{f}. The use of
GADTs allows us to statically prove that the \code{eval} function only ever
produces instances of type \code{b}.
%
The \code{codom} function allows one to find the type
descriptor associated to the return value (the codomain) of an \code{fn}.

Using the two functions above, it then becomes trivial to generate new instances
of \code{'b}.
%
% The function below doesn't need the special GADT syntax. Go figure.
\begin{ocamlcode}
let use (fd: ('a, 'b) fn) (f: 'a): () =
  let prod = eval fd f in
  let ty = codom fd in
  List.iter (fun x -> 
    if mem x ty then () else ty.enum <- x::ty.enum
  ) prod
\end{ocamlcode}
%
The function takes a function descriptor along with a matching function. The
\code{prod} variable contains all the instances of \code{'b} we just managed to
create; \code{ty} is the descriptor of \code{'b}. We store the new
instances of \code{'b} in the corresponding type descriptor.

In order to wrap this up nicely, one can define \emph{signature descriptors}. An
entry in a signature descriptor is merely a function of a certain
type \code{'a} along with its corresponding function descriptor. Once this is
done, the user can finally call our library and test the functions found in the
signature description.
%
\begin{ocamlcode}
type sig_elem = Elem : ('a,'b) fn * 'a -> sig_elem
type sig_descr = (string * sig_elem) list
let si_t  = (* create a descriptor for [SIList.t] and... *)
let int_t = (* one for [int], with a [fresh] function *)
let sig_of_silist = [
  ("empty", (returning si_t, SIList.empty));
  ("add", (int_t @-> si_t @-> returning si_t, SIList.add)); ]
let _ =
  Arti.generate sig_of_silist;
  assert (Arti.counter_example si_t SIList.check = None)
\end{ocamlcode}
%
The \code{Arti.generate} function repeatedly calls \code{use} on the items found in
the signature, until the desired number of instances have been created. We need
to perform several rounds of calls; with a single pass, the only applications we
would ever build would be of the form \code{add empty n}. We then check that the
\code{SIList.check} predicate holds for all the \code{SIList.t} generated.

% TODO: the transition from the introduction is a little bit dry, we probably
% need a few more paragraphs before that part.

\section{Implementing \arti}
\label{sec:representation}

The simplistic design we introduced in \sref{sec:essence} conveys the main ideas
behind \arti, yet fails to address a wide variety of cases. The present section
reviews the issues with the current design and incrementally addresses them.

\subsection{A better algebra of types}
\label{sec:algebra}

The simply-typed lambda calculus that we introduced only contains constants and
functions. While one can theoretically encode sums and products using functions,
it seems reasonable to have a built-in notion of sums and products in our
language.

One of the authors naïvely suggested that the data type be extended with cases
for products and sums, such as:
%
\begin{ocamlcode}
| Prod: ('a,'c) fn * ('b,'c) fn -> ('a * 'b,'c) fn
\end{ocamlcode}
%
It turns out that the branch above does not describe products. If \code{'a} is
\code{int -> int} and \code{'b} is \code{int -> float}, not only do the
\code{'c} parameters fail to match, but the \code{'a * 'b} parameter in the
conclusion represents a pair of functions, rather than a function that returns a
pair! Another snag is that the type of \code{eval} makes no sense in the case of
a product. If the first parameter of type \code{('a, 'b) fn} represents a way to
obtain a \code{'b} from the product type \code{'a}, then what use is the second
parameter of \code{eval}?

In light of these limitations, we take inspiration from the literature on
focusing and break the \code{fn} type into two distinct GADTs.
\begin{itemize}
  \item The GADT \code{('a, 'b) negative} (\code{neg} for short) represents a
    \emph{computation} of type \code{'a} that produces a result of type
    \code{'b}.
  \item The GADT \code{'a positive} (\code{pos} for short) represents a
    \emph{value}, that is, the result of a computation.
\end{itemize}
%
% 3a9c140
\begin{ocamlcode}
type (_, _) neg =
| Fun : 'a pos * ('b, 'c) neg -> ('a -> 'b, 'c) neg
| Ret : 'a pos -> ('a, 'a) neg

and _ pos =
| Ty : 'a ty -> 'a pos
| Sum : 'a pos * 'b pos -> ('a, 'b) sum pos
| Prod : 'a pos  * 'b pos -> ('a * 'b) pos
| Bij : 'a pos * ('a, 'b) bijection -> 'b pos

and ('a, 'b) sum =
| L of 'a
| R of 'b
\end{ocamlcode}
%
The \code{pos} type represents first-order data types: products, sums and atomic
types, that is, whatever is on the rightmost side of an arrow. We provide an
injection from positive to negative types: a value of type \code{'a} is also a
constant computation.

We do \emph{not} provide an injection from negative types to positive types:
this would allow nested arrows, that is, higher-order types.  One can take the
example of the \code{map} function, which has type \code{('a -> 'b) -> 'a list
-> 'b list}: we explicitly disallow representing the \code{'a -> 'b} part as a
\code{Fun} constructor, as it would require us to synthesize instances of a
function type. Rather, we ask the user to represent \code{'a -> 'b} as a
\code{Ty} constructor; in other words, we ask the user to supply their own test
functions as if they were a built-in type.

Since our GADT does not accurately model tagged, n-ary sums of OCaml, we provide
a last \code{Bij} case that allows the user to provide a two-way mapping between
a built-in type (say, \code{'a option}) and its \arti representation (\code{() +
'a}). That way, \arti can work with regular OCaml data types by converting them
back-and-forth.

This change of representation incurs some changes on our evaluation functions
as well. The \code{eval} function is split into several parts, which we detail
right below.
%
\begin{ocamlcode}
let rec apply: type a b. (a, b) neg -> a -> b list =
  fun ty v -> match ty with
  | Fun (p, n) ->
      produce p |> concat_map (fun a -> apply n (v a))
  ...
and produce: type a. a pos -> a list =
  fun ty -> match ty with
  | Ty ty -> Ty.elements ty
  | Prod (pa, pb) ->
      cartesian_product (produce pa) (produce pb)
  ...
let rec destruct: type a. a pos -> a -> unit =
  function
  | Ty ty -> (fun v ->
      remember v ty)
  | Prod (ta, tb) -> (fun (a, b) ->
      destruct ta a;
      destruct tb b)
  ...

(* Putting it all together *)
let _ =
  ...
  let li = apply fd f in
  List.iter (destruct head) li;
  ...
\end{ocamlcode}
%
Let us first turn to the case of \emph{values}. In order to understand what \arti
ought to do, one may ask themselves what the user can do with values. The user
may destruct them: given a pair of type \code{'a * 'b}, the user may keep just
the first element, thus obtaining a new \code{'a}. The same goes for sums. We
thus provide a \code{destruct} function, which breaks down positives types by
pattern-matching, populating the descriptions of the various types it encounters
as it goes.

Keeping this in mind, we must realize that if we can use a positive type
\code{'a} to obtain a \code{'b} (\code{apply}), the user may use any possible
means to produce an \code{'a}: if \code{'a} is a product, they will use every
possible combination of elements that are available to them; if \code{'a} is a
sum, they will either every choice from the type of either. We must therefore
devise a function \code{produce} that represents the entire set of possible
choices for a positive type.

The \code{apply} function, just like before, takes a \emph{computation} along
with a matching description, and generates a set of \code{b}. However, it now
relies on \code{product} to exhibit all possible instances of a type before
passing these instances on to the actual function.

We are now able to accurately model a calculus rich enough to test realistic
signatures involving records, option types, and various ways to create
functions.

\subsection{Efficient representation of a set of instances}
\label{sec:efficient}

At this stage, the set of features \arti offers is relatively satisfactory;
severe performance problems remain, however. Take the case of a module that
exports logical formulas, for instance. With only three combinators and twenty
constants, a first pass would generate 1200 formulas with one combinator. Using
these as an initial set, a second pass would generate more than 4,000,000
formulas, all of which would contain at most two combinators. In order to get to
interesting formulas that contain all three combinators, a third round of
combinations would be required, at which point the OCaml runtime crashes.

The (assuredly naïve) scenario above highlights two points.
\begin{itemize}
  \item One may wish for a better strategy, where the new instances generated by
    the first combinator are fed to the second combinator, thus making sure that
    by the time we reach the third combinator, at least \emph{some} instances
    contain all three combinators. We will see that this approach is not
    feasible, because of non-termination and a lack of fairness.
  \item One may also argue that we need a better representation for our sets of
    instances. This is indeed the case. In this section, we propose a symbolic
    algebra for \emph{sets of instances} that \emph{mirrors} the structure of
    positive types.
\end{itemize}
%
We first turn to the latter point, then tackle the issue of \emph{instance
propagation}.

The first, natural optimization that comes to mind consists in dropping lists in
favor of a more sophisticated data type. We replace lists with a module
\code{PSet} of persistent sets, implemented as red-black trees.

This is still not enough. A big source of inefficiency is the call to the
\code{cartesian\_product} function above (\sref{sec:algebra}). We hold in memory
at the same time all possible products, then pipe them into the function calls
so as to generate an even bigger set of elements. Only when the set of all
elements has been constructed do we actually run \code{destruct}, only to
extract the instances that we have created in the process.

Holding in memory the set of all possible products is too expensive. We adopt
instead a \emph{symbolic representation of sets}, where unions and products are
explicitly represented using constructors.
%
% 7dd765501e4d12d98e9d90a0ece490486cfe6913
\begin{ocamlcode}
type _ set =
  | Set   : 'a PSet.t -> 'a set
  | Bij   : 'a set * ('a, 'b) bijection -> 'b set
  | Union   : 'a set * 'b set -> ('a,'b) sum set
  | Product : 'a set * 'b set -> ('a * 'b) set
\end{ocamlcode}
%
This does not suppress the combinatorial explosion. The instance space is still
exponentially large; what we gained by changing our representation is that we
no longer hold all the ``intermediary'' instances in memory
\emph{simultaneously}. This allows us to write an \code{iter} function that
constructs the various instances on-the-fly.
%
\begin{ocamlcode}
  let rec iter: type a. (a -> unit) -> a set -> unit =
  fun f s -> match s with
    | Set ps ->
        PSet.iter f ps
    | Union (pa,pb) ->
        iter (fun a -> f (L a)) pa;
        iter (fun b -> f (R b)) pb;
    | Product (pa,pb) ->
        iter (fun a -> iter (fun b -> f (a,b)) pb) pa
    | (* ... *)
\end{ocamlcode}
%


\input{body}

\bibliographystyle{plain}
\hbadness=10000
\bibliography{local}

\end{document}
