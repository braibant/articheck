\documentclass[nonatbib]{sigplanconf}

\usepackage[T1]{fontenc}
\usepackage[utf8x]{inputenc}
\usepackage{minted}
\usepackage{xspace}


\newcommand{\arti}{\textsf{ArtiCheck}\xspace}
\newcommand{\qcheck}{QuickCheck\xspace}
\newcommand{\code}[1]{\texttt{#1}}
\newcommand{\sref}[1]{\S\ref{#1}}
\newminted{ocaml}{}

\begin{document}

% For ACM, explicitly specify A4 paper.
% For us, no need to specify it (but check your TeX configuration if
% you insist on producing a specific size).
\special{papersize=8.5in,11in}
\setlength{\pdfpageheight}{\paperheight}
\setlength{\pdfpagewidth}{\paperwidth}

\exclusivelicense

\conferenceinfo{ICFP~'14}{September 1--3, 2014, Copenhagen, Denmark}
\copyrightyear{2014}
\copyrightdata{978-1-nnnn-nnnn-n/yy/mm}
\doi{nnnnnnn.nnnnnnn}

\title{\arti: well-typed generic fuzzing for module interfaces}

\authorinfo{
  Thomas Braibant \quad
  Jacques-Henri Jourdan \quad
  Jonathan Protzenko \quad
  Gabriel Scherer
}{
  INRIA
}{http://gallium.inria.fr/blog/}
% \authorinfo{Thomas Braibant}
%            {INRIA}
%            {thomas.braibant@inria.fr}
% \authorinfo{Jacques-Henri Jourdan}
%            {INRIA}
%            {jacques-henri.jourdan@inria.fr}
% \authorinfo{Jonathan Protzenko}
%            {INRIA}
%            {jonathan.protzenko@ens-lyon.org}
% \authorinfo{Gabriel Scherer}
%            {INRIA}
%            {gabriel.scherer@inria.fr}

\maketitle

\begin{abstract}
  In spite of recent advances in full program certification, testing remains a
  widely-used component of the software development cycle. Various flavors of
  testing exist: popular ones include \emph{unit testing}, which consists in
  manually crafting test cases for specific parts of the code base, as well as
  \emph{quickcheck-style} testing, where instances of a type are automatically
  generated to serve as test inputs.

  These classical methods of testing can be thought of as \emph{internal}
  testing: the test routines access the internal representation of whatever data
  structure should be checked. We propose a new method of \emph{external}
  testing where the library only deals with a \emph{module interface}. The data
  structures are exported as \emph{abstract types}; the testing framework
  behaves just like regular client code and combines functions exported by the
  module to build new elements of the various types. Counter-examples, if any,
  are then presented to the user.
\end{abstract}

% TODO
\category{CR-number}{subcategory}{third-level}

\keywords{
functional programming,
testing,
quickcheck
}

\section{Introduction}
\label{sec:introduction}

Software development is hard. Industry practices still rely, for the better
part, on tests to ensure the functional correctness of programs. Even in more
sophisticated circles, such as the programming language research community, not
everyone has switched to writing all their programs in Coq. Testing is thus a
cornerstone of the development cycle. Moreover, even if the end goal is to fully
certify a program using a proof assistant, it is still worthwhile to eliminate
bugs early by running a cheap, efficient test framework.


Testing boils down to two different processes: generating test cases
for test suites; and then verifying that user-written assertions and
specifications of program parts are not falsified by the test suites.

\qcheck{}~\cite{DBLP:conf/icfp/ClaessenH00} is a popular, efficient
tool for that purpose. First, it provides a combinator library based
on type-classes to build test case generators. Second, it provides
a principled way for the users to specify properties over
functions. For instance, users may write predicates such as ``reverse
is an involution''. Then, the \qcheck framework is able to create
\emph{instances} of the type being tested, e.g., lists of integers.
The predicate is tested against these test cases, and any
counter-example is reported to the user.

Our novel approach is motivated by some limitations of the \qcheck
framework.  When users create trees, for instance, not only do they
have to specify that leaves should be generated more often than nodes
(for otherwise the tree generation would not terminate), but they also
have to rely on a global size measure to stop generating new nodes
after a while. It is thus up to the user of the library to implement
their own logic for generating the right instances, within a
reasonable size limit, combining the various base cases.

We argue that these low-level manipulations should be taken care of by the
library. When generating binary search tree instances, one ends up
re-implementing a series of random additions and deletions, which are precisely
the function that the code to be tested for exports. What if the testing
framework could, by itself, combine functions exported by the module we wish to
test, in order to build instances of the desired type? As long as the the module
exports its correctness properties, all the testing library needs is functions
that \emph{return} $t$'s.

In the present document, we describe a library that does precisely
that, dubbed \arti. The library is written in OCaml. While \qcheck
uses type-classes as a host language feature to conduct value
generation, we show how to implement the proof search in library
code -- while remaining both type-safe and generic over the
tested interfaces, thanks to GADTs.

\section{An example}

Consider the following signature, which describes the type of \emph{sorted}
integer lists.
%
\begin{ocamlcode}
module type SIList = sig
  type t

  val empty: t
  val add: t -> int -> t
  val invar: t -> bool
end
\end{ocamlcode}
%
The \code{invar} function dynamically checks that an element of type \code{t}
maintains the internal invariant. The module admits a straightforward
implementation, as follows.
%
\begin{ocamlcode}
module SIList = struct
  type t = int list

  let empty = []

  let rec add x = function
    | [] -> [x]
    | t::q -> if t<x then t::add x q else x::t::q

  let rec invar = function
    | [] -> true
    | [_] -> true
    | t1::(t2::_ as q) -> t1 <= t2 && invar q
end
\end{ocamlcode}

\section{Implementing \arti}
\label{sec:representation}

The simplistic design we introduced in \sref{sec:essence} conveys the main ideas
behind \arti, yet fails to address a wide variety of problems. The present section
reviews the issues with the current design and incrementally addresses them.

\subsection{A better algebra of types}
\label{sec:algebra}

The simply-typed lambda calculus that we introduced only contains constants and
functions. While one can theoretically encode sums and products using functions,
it seems reasonable to have a built-in notion of sums and products in our
language.

One of the authors naïvely suggested that the data type be extended with cases
for products and sums, such as:
%
\begin{ocamlcode}
| Prod: ('a,'c) fn * ('b,'c) fn -> ('a * 'b,'c) fn
\end{ocamlcode}
%
It turns out that the branch above does not describe products. If \code{'a} is
\code{int -> int} and \code{'b} is \code{int -> float}, not only do the
\code{'c} parameters fail to match, but the \code{'a * 'b} parameter in the
conclusion represents a pair of functions, rather than a function that returns a
pair! Another snag is that the type of \code{eval} makes no sense in the case of
a product. If the first parameter of type \code{('a, 'b) fn} represents a way to
obtain a \code{'b} from the product type \code{'a}, then what use is the second
parameter of \code{eval}?

In light of these limitations, we take inspiration from the literature on
% TODO \cite
focusing and break the \code{fn} type into two distinct GADTs.
\begin{itemize}
  \item The GADT \code{('a, 'b) negative} (\code{neg} for short) represents a
    \emph{computation} of type \code{'a} that produces a result of type
    \code{'b}.
  \item The GADT \code{'a positive} (\code{pos} for short) represents a
    \emph{value}, that is, the result of a computation.
\end{itemize}
%
% 3a9c140
\begin{ocamlcode}
type (_, _) neg =
| Fun : 'a pos * ('b, 'c) neg -> ('a -> 'b, 'c) neg
| Ret : 'a pos -> ('a, 'a) neg

and _ pos =
| Ty : 'a ty -> 'a pos
| Sum : 'a pos * 'b pos -> ('a, 'b) sum pos
| Prod : 'a pos  * 'b pos -> ('a * 'b) pos
| Bij : 'a pos * ('a, 'b) bijection -> 'b pos

and ('a, 'b) sum = L of 'a | R of 'b
\end{ocamlcode}
%
The \code{pos} type represents first-order data types: products, sums and atomic
types, that is, whatever is on the rightmost side of an arrow. We provide an
injection from positive to negative types via the \code{Ret} constructor: a
value of type \code{'a} is also a constant computation.

We do \emph{not} provide an injection from negative types to positive types:
this would allow nested arrows, that is, higher-order types.  One can take the
example of the \code{map} function, which has type \code{('a -> 'b) -> 'a list
-> 'b list}: we explicitly disallow
representing the \code{'a -> 'b} part as a
\code{Fun} constructor, as it would require us to synthesize instances of a
function type (see \sref{sec:higher-order} for a discussion).
Rather, we ask the user to represent \code{'a -> 'b} as a \code{Ty} constructor;
in other words, we ask the user to supply their own test functions as if they
were a built-in type.

Our GADT does not accurately model tagged, n-ary sums of OCaml, nor records with
named fields. We thus add a last \code{Bij} case; it allows the user to
provide a two-way mapping between a built-in type (say, \code{'a option}) and
its \arti representation (\code{() + 'a}). That way, \arti can work with regular
OCaml data types by converting them back-and-forth.

This change of representation incurs some changes on our evaluation functions
as well. The \code{eval} function is split into several parts, which we detail
right below.
%
\begin{ocamlcode}
let rec apply: type a b. (a, b) neg -> a -> b list =
  fun ty v -> match ty with
  | Fun (p, n) ->
      produce p |> concat_map (fun a -> apply n (v a))
  ...
and produce: type a. a pos -> a list =
  fun ty -> match ty with
  | Ty ty -> ty.enum
  | Prod (pa, pb) ->
      cartesian_product (produce pa) (produce pb)
  ...
let rec destruct: type a. a pos -> a -> unit =
  function
  | Ty ty -> (fun v ->
      remember v ty)
  | Prod (ta, tb) -> (fun (a, b) ->
      destruct ta a;
      destruct tb b)
  ...

(* Putting it all together *)
let _ =
  ...
  let li = apply fd f in
  List.iter destruct li;
  ...
\end{ocamlcode}
%
Let us first turn to the case of \emph{values}. In order to understand what \arti
ought to do, one may ask themselves what the user can do with values. The user
may destruct them: given a pair of type \code{'a * 'b}, the user may keep just
the first element, thus obtaining a new \code{'a}. The same goes for sums. We
thus provide a \code{destruct} function, which breaks down positives types by
pattern-matching, populating the descriptions of the various types it encounters
as it goes. (The \code{remember} function records all instances we haven't
encountered yet in the type descriptor \code{ty}.)

Keeping this in mind, we must realize that if a function takes an \code{'a}, the
user may use any \code{'a} it can produce to call the function. For instance, in
the case that \code{'a} is a product type \code{'a1 * 'a2}, then \emph{any} pair
of \code{'a1} and \code{'a2} may work.  We introduce a function called
\code{produce}, which reflects the fact the user may choose any possible pair:
the function exhibits the entire set of instances we can build for a given type.

Finally, the \code{apply} function, just like before, takes a \emph{computation}
along with a matching description, and generates a set of \code{b}. However, it
now relies on \code{product} to exhaustively exhibit all possible arguments one
can pass to the function.

We are now able to accurately model a calculus rich enough to test realistic
signatures involving records, option types, and various ways to create
functions.

\subsection{Efficient representation of a set of instances}
\label{sec:efficient}

The (assuredly naïve) scenario above reveals several pain points with the
current design.
\begin{itemize}
  \item We represent our sets using lists. We could use a more efficient data
    structure.
  \item If some function takes, say, a tuple, the code as it stands will
    construct the set of all possible tuples, \code{map} the function over the
    set, then finally call \code{destruct} on each resulting element to collect
    instances.  Naturally, memory explosion ensues. We propose a symbolic
    algebra for \emph{sets of instances} that \emph{mirrors} the structure of
    positive types and avoids the need for holding all possible combinations in
    memory at the same time.
  \item A seemingly trivial optimization sends us off the rails by generating an
    insane number of instances. We explain how to optimize further the code
    while still retaining a well-behaved generation.
  \item Fairness issues arise.
    Take the example of logical formulas. One may try to be smart: starting with
    constants, one may apply \code{mk\_and}, then pass the freshly generated
    instances to \code{mk\_xor}. A consequence is that all the formulas with two
    combinators start with \code{xor}. If we just keep an iterative process and
    do not chain the instance generation process, formulas containing three
    combinators are only reached after we've exhausted all possible instances
    with two or less combinators. This breadth-first search of the instance
    space is sub-optimal. Can we do better?
\end{itemize}
%

\paragraph{Sets of instances}
The first, natural optimization that comes to mind consists in dropping lists in
favor of a more sophisticated data type. We replace lists with a module
\code{PSet} of polymorphic, persistent sets implemented as red-black trees.


\paragraph{Not holding sets in memory}
A big source of inefficiency is the call to the
\code{cartesian\_product} function above (\sref{sec:algebra}). We hold in memory
at the same time all possible products, then pipe them into the function calls
so as to generate an even bigger set of elements. Only when the set of all
elements has been constructed do we actually run \code{destruct}, only to
extract the instances that we have created in the process.

Holding in memory the set of all possible products is too expensive. We adopt
instead a \emph{symbolic representation of sets}, where unions and products are
explicitly represented using constructors. This mirrors our algebra of positive
types.
%
% 29bab9d5d66f066906b5b3d1449fd02cf34aa7dc
\begin{ocamlcode}
type _ set =
  | Set   : 'a PSet.t -> 'a set
  | Bij   : 'a set * ('a, 'b) bijection -> 'b set
  | Union   : 'a set * 'b set -> ('a,'b) sum set
  | Product : 'a set * 'b set -> ('a * 'b) set
\end{ocamlcode}
%
This does not suppress the combinatorial explosion. The instance space is still
exponentially large; what we gained by changing our representation is that we
no longer hold all the ``intermediary'' instances in memory
\emph{simultaneously}. This allows us to write an \code{iter} function that
constructs the various instances on-the-fly.
%
\begin{ocamlcode}
let rec iter: type a. (a -> unit) -> a set -> unit =
fun f s -> match s with
  | Set ps ->
      PSet.iter f ps
  | Union (pa,pb) ->
      iter (fun a -> f (L a)) pa;
      iter (fun b -> f (R b)) pb;
  | Product (pa,pb) ->
      iter (fun a -> iter (fun b -> f (a,b)) pb) pa
  | (* ... *)
\end{ocamlcode}
%

\paragraph{Piping and non-termination}
In order to push the optimization above further, one can choose to perform the
call to \code{remember} directly inside the \code{Ret} case of \code{apply}.
That way, \code{apply} could just fill in the type descriptors using the global,
mutable state and return unit, thus avoiding the need for intermediary lists of
instances. Also, calling \code{remember} directly eliminates the need to store
duplicate items, as the function automatically takes care of dropping an
instance if we are already aware of it.

This seemingly innocuous optimization raised combinatorial explosion issues. We
explain why, in the hope that it serves as an example for future generations
(``kids, don't do mutable state'').

Consider the case of a function that has type \code{t -> t -> t} and a
corresponding type descriptor for \code{t} named \code{ty}. The outer call
to \code{apply} binds the list of instances of \code{t} via \code{let l =
ty.enum}. For each element of \code{l}, a recursive call to \code{apply} takes
place (for the inner \code{t -> t} function), which looks up the current value
of \code{ty.enum}. Since each inner call populates \code{ty.enum} itself, for
each new recursive call of \code{apply}, the value of \code{ty.enum} grows
bigger and bigger. The programs terminates by exhausting its memory space
without even returning from the outer call to \code{apply}.

We solved this by taking a snapshot of our negative types before calling
\code{apply}. No copy is involved: function arguments (positive types) are
represented in memory as persistent, pure symbolic sets. That way, we keep a
copy of the arguments that are to be applied in each \code{Fun} case.

\paragraph{Fairness of our search space}
% TODO jp: we have to redo this part entirely (see my recap in the itemize
% section at the beginning)
Snapshotting enforces a breadth-first search of the instance space. The initial
set of instances is fed through the available functions, and we iterate the
process, until we've obtained a satisfactory number of instances for each one of
the types we wish to test.

The distribution of instances is skewed: there are more instances obtained after
\code{n} calls than there are after \code{n+1} calls. It may thus be the case
that by the time we reach three of four consecutive function calls, we've hit
the maximum limit of instances allowed for the type, since it often is the case
that the number of instances grow exponentially.

We plan to implement a random search of the instance space and tweak our
exploration procedures so that ``interesting'' instances pop up early.

\subsection{Instance generation as a fixed point computation}

The \code{apply}/\code{destruct} combination only demonstrates how to generate
new instances from one specific element of the signature. We need to iterate
this recipe on the whole signature, by feeding the new instances that we
obtain to other functions that can in turn consume them.

This part of the problem naturally presents itself as a fixpoint computation,
defined by a system of equations. Equations between variables (type descriptors)
describe ways of obtaining new instances (by applying functions to other type
descriptors). Of course, to ensure termination, we need to put a bound on the
number of generated instances. When presenting an algorithm as a fixpoint
problem, it is indeed a fairly standard technique to make the lattice space
artificially finite in order to obtain the termination property.

Implementing an efficient fixpoint computation is a \emph{surprisingly
interesting} activity, and we are happy to use an off-the-shelf fixpoint
library, François Pottier's \code{Fix},
%TODO href?
to perform the work for us. \code{Fix} can be summarized by the signature
below, obtained from user-defined instantiations of the types \code{variable}
and \code{property}.
%
\begin{ocamlcode}
module Fix = sig
  type valuation = variable -> property
  type rhs = valuation -> property
  type equations = variable -> rhs

  val lfp: equations -> valuation
end
\end{ocamlcode}
%
% TODO \cite
A system of equations maps a variable to a right-hand side.  Each right-hand
side can be evaluated by providing a valuation so as to obtain a property.
Valuations map variables to properties. Solving a system of equations amounts to
calling the \code{lfp} function which, given a set of equations, returns the
corresponding valuation.

A perhaps tempting way to fit in this setting would be to define variables to be
our \code{'a ty} (type descriptor) and properties to be \code{'a list}s (the
instances we have built so far); the equations derived from any signature would
then describe ways of obtaining new instances by applying any function of the
signature. This doesn't work as is: since there will be multiple values of
\code{'a} (we generate instances of different types simultaneously), type
mismatches are to be expected. One could, after all, use yet another GADT and
hide the \code{'a} type parameter behind an existential variable.
%
\begin{ocamlcode}
  type variable = Atom: 'a ty -> variable
  type property = Props: 'a set -> property
\end{ocamlcode}
%
The problem is that there is no way to statically prove that having an
\code{'a var} named \code{x}, calling \code{valuation x} yields an
\code{'a property} with a matching type parameter. This is precisely
where the mutable state in the \code{'a ty} type comes handy: even though it is
only used as the \emph{input} parameter for the system of equations, we
``cheat'' and use its mutable \code{enum} field to store the output. That way,
the \code{property} type needs not mention the type variable \code{'a} anymore,
thus removing any typing difficulty -- or the need to change the interface of
\code{Fix}.

We still, however, need the \code{property} type to be a rich enough lattice to
let \code{Fix} decide when to stop iterating: it should come with equality- and
maximality-checking functions, used by \code{Fix} to detect that the fixpoint is
reached. The solution is to define \code{property} as the number of instances
generated so far along with the bound we have chosen in advance:
%
\begin{ocamlcode}
  type variable = Atom : 'a ty -> variable
  type property = { required : int; produced : int }
  let equal p1 p2 = p1.produced = p2.produced
  let is_maximal p = p.produced >= p.required
\end{ocamlcode}
%

\section{Expressing correctness properties}
\label{sec:properties}

We mentioned earlier (\sref{sec:essence}) the \code{counter\_example} function.
% TODO: remove the signature?
%
\begin{ocamlcode}
val counter_example: 'a pos -> ('a -> bool) -> 'a option
\end{ocamlcode}
%
The function takes a description of some (positive) datatype
\code{'a}, iterates on the generated instances of this type and checks
that a predicate \code{'a -> bool} holds for all instances, or returns
a counter-example otherwise. At a more abstract level, this means that we are
checking a property of the form $ \forall (x \in t), T(x) $ where
$T(x)$ is simply a boolean expression.
Multiple quantifiers can be simulated through the use of product types, such as
in the typical formula of association maps:
%
\[
\forall (m \in \mathtt{map}(K,V), k \in K, v \in V),
  \ \mathtt{find(k,add(k,v,m)) = v}
\]
%
which can be expressed as follows (where \code{*@} is the operator for creating
product type descriptors):
%
\begin{ocamlcode}
let lookup_insert_prop (k, v, m) =
  lookup k (insert k v m) = v
let () = assert (None =
  let kvm_t = k_t *@ v_t *@ map_t in
  counter_example kvm_t lookup_insert_prop)
\end{ocamlcode}

One then naturally wonders what a good language would be for describing the
correctness properties we wish to check. In the example above, we naturally
veered towards first-order logic, so as to express formulas with only prenex,
universal quantification. The universal quantifiers are to be understood with a
``test semantics'', that is, they mean to quantify over all the random instances
we generated.
%
Can we do better? In particular, can we capture the full language of first-order
logic, as a reasonable test \emph{description language} for a practical
framework?

It feels natural to use first-order logic as a specification language
in the context of structured verification, such as with SMT solvers or
a finite model finder~\cite{DBLP:conf/itp/BlanchetteN10}.  However,
supporting full first-order logic as a specification language for
randomly-generated tests is hard for various reasons.

For instance, giving ``test semantics'' to an existentially-quantified formula
such as $\exists(x \in t). T(x)$ is awkward. Intuitively, there is not much
meaning to the formula. The number of generated instances is finite; that none
satisfies $T$ may not indicate a bug, but rather that the wrong elements have
been tested for the property. Conversely, finding a counter-example to a
universally-quantified formula \emph{always} means that a bug has been found.
%
Trying to distinguish absolute (positive or negative) results from probabilistic
results opens a world of complexity that we chose not to explore.

Surprisingly enough, there does not seem to be a consensus in the
literature about random testing for an expressive, well-defined subset of
first-order logic. The simplest subset one directly thinks of is
formulas of the form:
$ \forall x_1 \dots x_n, P(x_1, \dots, x_n) \Rightarrow T(x_1, \dots,
x_n) $ where $P(x_1, \dots, x_n)$ (the \emph{precondition}) and
$T(x_1, \dots, x_n)$ (the \emph{test}) are both quantifier-free
formulas.

The reason this implication is given a specific status is to make it possible to
distinguish tests that succeeded because the \emph{test} was effectively
successful from tests that succeeded because the \emph{precondition} was not
met. The latter are ``aborted'' tests that bring no significant value, and
should thus be disregarded. In \arti, we chose to restrict ourselves to this
last form of formulas.

\section{Examples}

\subsection{Red-black trees}
The (abridged) interface exported by red-black trees is as follows. The module
provides iteration facilities over the tree structure through the use of
\cite{huet-zipper-97}
\emph{zippers}. Our data structures are persistent.
%
\begin{ocamlcode}
module type RBT = sig
  type 'a t

  val empty : 'a t
  val insert : 'a -> 'a t -> 'a t

  type direction = Left | Right
  (* type 'a zipper *)
  type 'a ptr (* = 'a t * 'a zipper *)

  val zip_open : 'a t -> 'a ptr
  val zip_close : 'a ptr -> 'a t

  val move_up : 'a ptr -> 'a ptr option
  val move : direction -> 'a ptr -> 'a ptr option
end
\end{ocamlcode}
%
This examples highlights several strengths of \arti.

First, two different types are involved: the type of trees and the type of
zippers. While an aficionado of internal testing may use the \code{empty} and
\code{insert} functions repeatedly to create new instances of \code{'a t}, it
becomes harder to type-check calls to \emph{either} \code{insert} or
\code{zip\_open}. Our framework, thanks to GADTs, generates instances of both
types painlessly and automatically.

Second, we argue that a potential mistake is detected trivially by \arti, while
it may turn out to be harder to detect using internal testing. If one removes
the comments, the signature reveals that pointers into a tree are made up of a
zipper along with a tree itself. It seems fairly natural that the developer
would want to reveal the \code{zipper} type; it is, after all, a fundamental
feature of the module. An undercaffeinated developer, when writing internal test
functions, would probably perform sequences of calls to the various functions.
What they would fail to do, however, is destructing pairs so as to produce
a zipper associated with \emph{the wrong tree}. This particularly wicked usage
would probably be overlooked. \arti successfully destructs the pair and performs
recombinations, to finally output:
\begin{verbatim}
  TODO: fix the code so that it terminates
    ... and put the error message here
\end{verbatim}

\subsection{Binary Decision Diagrams}

Binary Decision Diagrams (BDDs) represent trees for deciding logical formulas. The
defining characteristic of BDDs is that they enforce \emph{maximal sharing}:
wherever two structurally equal sub-formulas appear, they are guaranteed to
refer to the same object in memory. A consequence is that performing large
numbers of function calls does not necessarily means using substantially more
memory: it may very well be the case that significant sharing occurs.

We mentioned earlier that our strategy for external testing amounted, in
essence, to representing series of well-typed function calls in the simply typed
lambda calculus using in GADT. If we only did that and skipped section
\sref{sec:representation}, externally-testing BDDs would be infeasible, as we
would end up representing a huge number of function calls in memory.

Conversely, with the design we exposed earlier, we merely record new instances
as they appear without holding the entire set of potential function calls in
memory. This allows for an efficient, non-redundant generation of test cases
(instances).

\subsection{AVL trees}

AVL trees are a classic of programming interviews; many a graduate
student has been scared by the mere mention of them. It turns out that tenured
professors \emph{should} be scared too: the OCaml implementation of maps,
written using AVL trees by a respectable researcher, contained a bug that went
unnoticed for more than ten years.
% TODO fact checking
The bug was discovered when another enthusiastic researcher set out to formalize
the said library in Coq. The bug was fixed, and all was well. Out of curiosity,
we decided to run \arti on the faulty version of the library.
% TODO wrote that in anticipation, so fix afterwards KTHXBYE
After registering only four functions with \arti, the bug was correctly
identified by our library, with arguably less pain than the full Coq
formalization required.



\section{Related and Future Work}

\paragraph{Genericity of value generation}

The idea of generating random sequences of operations instead of
random internal values is not novel; for example, \qcheck was used as
is to test imperative
programs~\cite{DBLP:journals/sigplan/ClaessenH02}, by generating
random values of an AST of operations, paired to a monadic interpreter
of those syntactic descriptions. However, those examples in the
literature only involve operations for a single return type,
corresponding to the return type of the AST evaluation function. To
integrate operations of distinct return types in the same interface
description, one needs GADTs or some other form of type-level
reasoning.

When multiple value types are involved, we found it helpful to think
of well-typed value generation as term/proof search. Our well-typed
rule to generate random values at type $\tau$ from a function at type
$\sigma \to \tau$ and random values at type $\sigma$ could be
expressed, in term of \qcheck \code{Arbitrary} instances, as
a deduction rule of the form \code{instance Arbitrary b, Arbitrary
  (a -> b) => Arbitrary b where ...}. However, Haskell's type-class
mechanism would not allow this instance deduction rule, as it does not
respect its restrictions for principled, coherent instance
elaboration. Type classes are a helpful and convenient host-language
mechanism, but they are designed for code inference rather than
arbitrary proof search. Our library-level implementation of well-typed
proof search using GADTs gives us more freedom, and is the central
idea of \arti.

It is of course possible to see chaining of function/method calls as
a metaprogramming problem, and generate a reified description of those
calls, to interpret through an external script or reflection/JIT
capability, as done in the testing tool
Randoop~\cite{DBLP:conf/oopsla/PachecoE07}. Doing the generation as
a richly-typed host language library gives us stronger type safety
guarantees: even if our value generator is buggy, it will never
compose operations in a type-incorrect way.

\paragraph{Testing of higher-order or polymorphic interfaces}
\label{sec:higher-order}

The type description language we use captures a first-order subset of
the simply-typed lambda-calculus. A natural question is whether it
would be possible to support random function generation -- embed
negative types into positives. A simple way to generate a function of
type \code{a -> b} is to just generate a \code{b} at each call;
\qcheck additionally uses the \code{a} argument to produce additional
entropy. This is not completely satisfying as it does not use the
argument at type \code{a} (which may not be otherwise reachable from
the interface) to produce new values of type \code{b}. To have full
test coverage for higher-order functional, one should locally add the
argument to the set of known elements at type \code{a}, and
re-generate values at type \code{b} in that extended environment.

It would also be interesting to support representation of polymorphic
operations; we currently only describe monomorphic
instantiations. Bernardy \emph{et. al.}~\cite{DBLP:conf/esop/BernardyJC10} have proposed
a parametricity-based technique to derive specific monomorphic
instances for type arguments, which also reduces the search space of
values to be tested. Supporting this technique would be a great asset
of a testing library, but it is definitely not obvious how their
pen-and-paper derivation could be automatized, especially as a library
function.

\paragraph{Bottom-up or top-down generation}

We have presented the \code{ArtiCheck} implementation as a bottom-up
process: from a set of already-discovered values at the types of
interest, we use the constructors of the interface to produce new
values. In contrast, most random checking tools present generation in
a top-down fashion: pick the head constructor of the data value, then
generate its subcomponents recursively. One notable exception is
SmallCheck~\cite{DBLP:conf/haskell/RuncimanNL08}, which performs
exhaustive testing for values of bounded depth.

The distinction is however blurred by several factors. Fix implements
demand-driven computation of fixpoints: if you request elements at
type \code{u} and there is an operation \code{t -> u}, it will
recursively populate values at type \code{t}, giving the actual
operational behavior of the generator a top-down flavor. Relatedly,
SmallCheck has a Lazy SmallCheck variant that uses laziness
(demand-driven computation) to avoid fleshing out value parts that are
not inspected by the property being tested.

Furthermore, the genericity of our high-level interface makes
\code{Articheck} amenable to a change in the generation technique; we
could implement direct top-down search without changing the signature
description language, or most parts of the library interface.

\paragraph{Richer property languages}

We discussed in Section~\ref{sec:properties} the difficulty of
isolating an expressive fragment of first-order logic as a property
language that could be given a realizable testing semantics. As it
performs exhaustive search (up to a bound), SmallCheck is able to give
a non-surprising semantics to existential quantification. As we let
user control for each interface datatype whether an exhaustive
collection or a sampling collection should be used, we could
support existential on exhaustively collected types only.

In related manner, Berghofer and Nipkow's
implementation of QuickCheck for
Isabelle~\cite{DBLP:conf/sefm/BerghoferN04} stands out by supporting full-fledged
first-order logic for random checking. In the Isabelle proof
assistant, it is common to define computations as inductive
relations/predicates that can be given a (potentially
non-deterministic) functional mode; instead of directly turning
correctness formulas into testing programs, they translate formulas
into inductive datatypes, which are then given a computational
interpretation.

This is remarkable as it not only allows them to support a rich
specification language, but also gives a principled explanation for
the ad-hoc semantics of preconditions in testing frameworks (a failing
precondition does not count as a passed test); instead of seeing
a precondition \code{P[x]} as returning a boolean from
a randomly-generated \code{x}, they choose a mode assignment that
inverts it into a logic program generating the \code{x}s accepted by
the precondition. This gives a logic-based justification to various
heuristics used in other works to generate random values more likely
to pass the precondition, either
domain-specific~\cite{DBLP:conf/tap/ClaessenS08} or
SAT-based~\cite{DBLP:conf/tap/AhnD10}.

\section{Conclusion}

We have presented the design of \arti, a novel library that allows one to check
the invariants of a module signature by simulating user interaction with the
module. \arti behaves like a fake client: it calls functions, constructs and
destructs products or sums, and for each element check that the invariants are
verified. The key to performing this in a generic, abstract manner relies on
GADTs, which abstract the different types that may be manipulated into a common
representation.

We identified various performance problems that arise. The library handles them
via a symbolic representation of types in combination with a little bit of
mutable state to avoid handling large, intermediary results in memory.

The result is a self-contained library that wraps the core concepts of
\emph{external testing} and offers clients a cheap and efficient way to test
their programs. The library, for instance, successfully detects infamous issues
such as the AVL re-balancing issue in the standard library of OCaml, with a much
lower cost than a complete Coq proof of the module.

While the library exposes the essence of \emph{external testing} and has already
proven worthwhile, we believe there is potential for improvement and expansion
into a fully-fledged testing library.

% TODO: related work starts here
Bernardy et al.~\cite{DBLP:conf/esop/BernardyJC10} describe a
systematic way of reducing the testing of polymorphic functions to the
testing of specific monomorphic instances of these functions.
%
Given a polymorphic property, the correctness of the reduced
(monomorphic) property entails the correctness of all other
instanciations. This yields a significant reduction in the necessary
test cases.
% They rely on a version of parametricity that does not hold in ML in
% all its generality. For instance, one should not use side effects.
%
They informally argue that their technique is efficient compared to
the standard praxis of substituting \texttt{int} for polymorphic
types.
%
Note however that both solutions to the problem of testing polymorphic
functions must be applied at the meta-level. That is, the user has to
pick the right instanciation of polymorphic type variables; this
cannot be done automatically inside the host language.


\bibliographystyle{plain}
\hbadness=10000
\bibliography{local}

\end{document}
