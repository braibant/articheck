\documentclass[nonatbib]{sigplanconf}

\usepackage[T1]{fontenc}
\usepackage[utf8x]{inputenc}

\begin{document}

% For ACM, explicitly specify A4 paper.
% For us, no need to specify it (but check your TeX configuration if
% you insist on producing a specific size).
\special{papersize=8.5in,11in}
\setlength{\pdfpageheight}{\paperheight}
\setlength{\pdfpagewidth}{\paperwidth}

\exclusivelicense

\conferenceinfo{ICFP~'14}{September 1--3, 2014, Copenhagen, Denmark}
\copyrightyear{2014}
\copyrightdata{978-1-nnnn-nnnn-n/yy/mm} 
\doi{nnnnnnn.nnnnnnn}

\title{Functional pearl: zero-knowledge testing for module interfaces}

\authorinfo{
  Thomas Braibant \quad
  Jacques-Henri Jourdan \quad
  Jonathan Protzenko \quad
  Gabriel Scherer
}{
  INRIA
}{http://gallium.inria.fr/blog/}
% \authorinfo{Thomas Braibant}
%            {INRIA}
%            {thomas.braibant@inria.fr}
% \authorinfo{Jacques-Henri Jourdan}
%            {INRIA}
%            {jacques-henri.jourdan@inria.fr}
% \authorinfo{Jonathan Protzenko}
%            {INRIA}
%            {jonathan.protzenko@ens-lyon.org}
% \authorinfo{Gabriel Scherer}
%            {INRIA}
%            {gabriel.scherer@inria.fr}

\maketitle

\begin{abstract}
  In spite of recent advances in full program certification, testing remains a
  widely-used component of the software development cycle. Various flavors of
  testing exist: popular ones include \emph{unit testing}, which consists in
  manually crafting test cases for specific parts of the code base, as well as
  quickcheck-style testing, where instances of a type are automatically
  generated to serve as test inputs.

  These classical methods of testing can be thought of as \emph{internal}
  testing: the test routines access the internal representation of whatever
  module should be checked. We propose a new method of \emph{external} testing
  where test code checks an \emph{abstract} data structure. Our new testing
  method takes a description of a \emph{module signature}, then builds sequences
  of function calls that generate elements of the abstract type just like any other
  client code. Our testing framework thus builds elements that respect the
  invariants hidden by the module signature.
\end{abstract}

\category{CR-number}{subcategory}{third-level}

\keywords{
functional programming,
testing,
quickcheck
}

\bibliographystyle{plain}
\hbadness=10000
\bibliography{local}

\end{document}
