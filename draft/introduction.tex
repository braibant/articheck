\section{Introduction}

\newcommand\qcheck{QuickCheck~}

Software development is hard. Industry practices still rely, for the better
part, on tests to ensure the functional correctness of programs. Even in more
sophisticated circles, such as the programming language research community, not
everyone has switched to writing all their programs in Coq. Testing is thus a
cornerstone of the development cycle. Moreover, even if the end goal is to fully
certify a program using a proof assistant, it is still worthwhile to eliminate
bugs early by running a cheap, efficient test framework.


Testing boils down to two different processes: generating test cases
for test suites; and then verifying that user-written assertions and
specifications of program parts are not falsified by the test suites.

\qcheck{} is a popular, efficient tool for that purpose. First, it
provides a combinator library based on type-classes to build test case
generators. Second, it provides a principled way for the users to
specify properties over functions. For instance, users may write
predicates such as ``reverse is an involution''. Then, the \qcheck
framework is able to create \emph{instances} of the type being tested,
e.g., lists of integers.  The predicate is tested against these test
cases, and any counter-example is reported to the user.

Our novel approach is motivated by some limitations of the \qcheck
framework.  When users create trees, for instance, not only do they
have to specify that leaves should be generated more often than nodes
(for otherwise the tree generation would not terminate), but they also
have to rely on a global size measure to stop generating new nodes
after a while. It is thus up to the user of the library to implement
their own logic for generating the right instances, within a
reasonable size limit, combining the various base cases.

We argue that these low-level manipulations should be taken care of by the
library. When generating binary search tree instances, one ends up
re-implementing a series of random additions and deletions, which are precisely
the function that the code to be tested for exports. What if the testing
framework could, by itself, combine functions exported by the module we wish to
test, in order to build instances of the desired type? As long as the the module
exports a correctness predicate, all the testing library needs is functions
that \emph{return} $t$'s.

In the present document, we describe a library that does precisely that, dubbed
``articheck''. The library is written in OCaml. While \qcheck uses a
combination internal testing and type classes, our library performs external
testing and relies on GADTs.