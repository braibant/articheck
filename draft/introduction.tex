\section{Introduction}
\label{sec:introduction}

Software development is hard. Industry practices still rely, for the better
part, on tests to ensure the functional correctness of programs. Even in more
sophisticated circles, such as the programming language research community, not
everyone has switched to writing all their programs in Coq. Testing is thus a
cornerstone of the development cycle. Moreover, even if the end goal is to fully
certify a program using a proof assistant, it is still worthwhile to eliminate
bugs early by running a cheap, efficient test framework.

Testing boils down to two different processes: generating test cases
for test suites; then verifying that user-written assertions and
specifications of program parts are not falsified by the test suites.

\qcheck{}~\cite{DBLP:conf/icfp/ClaessenH00} is a popular, efficient
tool for that purpose. First, it provides a combinator library based
on type-classes to build test case generators. Second, it provides
a principled way for users to specify properties over
functions. For instance, users may write predicates such as ``reverse
is an involution''. Then, the \qcheck framework is able to create
\emph{instances} of the type being tested, e.g. lists of integers.
The predicate is tested against these test cases, and any
counter-example is reported to the user.

Our novel approach is motivated by some limitations of the \qcheck
framework.
%
First, data-structures often have \emph{internal invariants} that
would not be broken when using the API of the data-structure. Thus,
testing one particular function of such an API requires generating
test-cases that meet these invariants.
%
Yet, writing \emph{good} generators for (involved) data-structures is
tedious, or even plain hard. Consider the simple problem of generating
binary search trees (BST), for instance. Being smarter than merely
generating trees and filtering out those which are not
search trees requires reimplementing a series of insertions and
deletions into BSTs. But these functions are most certainly already
part of the code that is tested!
%

We argue that this low-level manipulation could be taken care of by
the testing framework. That is, we argue that the framework should be
able, by itself, to combine functions exported by the module we wish
to test in order to build instances of the data-types defined in the
same module. If the module exports the properties and the invariants
that should hold, then the testing framework needs not see the
implementation.
%
In a nutshell, we want to move towards an \emph{external testing} of abstract
modules.

In the present document, we describe a library that does precisely
that, dubbed \arti. The library is written in OCaml. While \qcheck
uses type-classes as a host language feature to conduct value
generation, we show how to implement the proof search in library
code -- while remaining both type-safe and generic over the
tested interfaces, thanks to GADTs.
