\section{Examples}

\subsection{Red-black trees}
The (abridged) interface exported by red-black trees is as follows. The module
provides iteration facilities over the tree structure through the use of
\cite{huet-zipper-97}
\emph{zippers}. Our data structures are persistent.
%
\begin{ocamlcode}
module type RBT = sig
  type 'a t

  val empty : 'a t
  val insert : 'a -> 'a t -> 'a t

  type direction = Left | Right
  (* type 'a zipper *)
  type 'a ptr (* = 'a t * 'a zipper *)

  val zip_open : 'a t -> 'a ptr
  val zip_close : 'a ptr -> 'a t

  val move_up : 'a ptr -> 'a ptr option
  val move : direction -> 'a ptr -> 'a ptr option
end
\end{ocamlcode}
%
This examples highlights several strengths of \arti.

First, two different types are involved: the type of trees and the type of
zippers. While an aficionado of internal testing may use the \code{empty} and
\code{insert} functions repeatedly to create new instances of \code{'a t}, it
becomes harder to type-check calls to \emph{either} \code{insert} or
\code{zip\_open}. Our framework, thanks to GADTs, generates instances of both
types painlessly and automatically.

Second, we argue that a potential mistake is detected trivially by \arti, while
it may turn out to be harder to detect using internal testing. If one removes
the comments, the signature reveals that pointers into a tree are made up of a
zipper along with a tree itself. It seems fairly natural that the developer
would want to reveal the \code{zipper} type; it is, after all, a fundamental
feature of the module. An undercaffeinated developer, when writing internal test
functions, would probably perform sequences of calls to the various functions.
What they would fail to do, however, is destructing pairs so as to produce
a zipper associated with \emph{the wrong tree}. This particularly wicked usage
would probably be overlooked. \arti successfully destructs the pair and performs
recombinations, thus finding the bug.

\subsection{Binary Decision Diagrams}

Binary Decision Diagrams (BDDs) represent trees for deciding logical formulas. The
defining characteristic of BDDs is that they enforce \emph{maximal sharing}:
wherever two structurally equal sub-formulas appear, they are guaranteed to
refer to the same object in memory. A consequence is that performing large
numbers of function calls does not necessarily means using substantially more
memory: it may very well be the case that significant sharing occurs.

We mentioned earlier that our strategy for external testing amounted, in
essence, to representing series of well-typed function calls in the simply typed
lambda calculus using in GADT. If we only did that and skipped section
\sref{sec:representation}, externally-testing BDDs would be infeasible, as we
would end up representing a huge number of function calls in memory.

Conversely, with the design we exposed earlier, we merely record new instances
as they appear without holding the entire set of potential function calls in
memory. This allows for an efficient, non-redundant generation of test cases
(instances).

\subsection{AVL trees}

AVL trees are a classic of programming interviews; many a graduate
student has been scared by the mere mention of them. It turns out that
tenured professors \emph{should} be scared too: the OCaml
implementation of sets, written using AVL trees by a respectable
researcher, contained a bug that went unnoticed for more than ten
years. The bug was discovered when another enthusiastic researcher
set out to formalize the said library in Coq. The bug was fixed, and
all was well. Out of curiosity, we decided to run \arti on the faulty
version of the library.  After registering only four functions with
\arti, the bug was correctly identified by our library, with arguably
less pain than the full Coq formalization required.
