% TODO: the transition from the introduction is a little bit dry, we probably
% need a few more paragraphs before that part.

\section{Implementing \arti}
\label{sec:representation}

The simplistic design we introduced in \sref{sec:essence} conveys the main ideas
behind \arti, yet fails to address a wide variety of cases. The present section
reviews the issues with the current design and incrementally addresses them.

\subsection{A better algebra of types}
\label{sec:algebra}

The simply-typed lambda calculus that we introduced only contains constants and
functions. While one can theoretically encode sums and products using functions,
it seems reasonable to have a built-in notion of sums and products in our
language.

One of the authors naïvely suggested that the data type be extended with cases
for products and sums, such as:
%
\begin{ocamlcode}
| Prod: ('a,'c) fn * ('b,'c) fn -> ('a * 'b,'c) fn
\end{ocamlcode}
%
It turns out that the branch above does not describe products. If \code{'a} is
\code{int -> int} and \code{'b} is \code{int -> float}, not only do the
\code{'c} parameters fail to match, but the \code{'a * 'b} parameter in the
conclusion represents a pair of functions, rather than a function that returns a
pair! Another snag is that the type of \code{eval} makes no sense in the case of
a product. If the first parameter of type \code{('a, 'b) fn} represents a way to
obtain a \code{'b} from the product type \code{'a}, then what use is the second
parameter of \code{eval}?

In light of these limitations, we take inspiration from the literature on
focusing and break the \code{fn} type into two distinct GADTs.
\begin{itemize}
  \item The GADT \code{('a, 'b) negative} (\code{neg} for short) represents a
    \emph{computation} of type \code{'a} that produces a result of type
    \code{'b}.
  \item The GADT \code{'a positive} (\code{pos} for short) represents a
    \emph{value}, that is, the result of a computation.
\end{itemize}
%
\begin{ocamlcode}
type (_, _) neg =
| Fun : 'a pos * ('b, 'c) neg -> ('a -> 'b, 'c) neg
| Ret : 'a pos -> ('a, 'a) neg

and _ pos =
| Ty : 'a ty -> 'a pos
| Sum : 'a pos * 'b pos -> ('a, 'b) sum pos
| Prod : 'a pos  * 'b pos -> ('a * 'b) pos
| Bij : 'a pos * ('a, 'b) bijection -> 'b pos

and ('a, 'b) sum =
| L of 'a
| R of 'b
\end{ocamlcode}
%
The \code{pos} type represents first-order data types: products, sums and atomic
types, that is, whatever is on the rightmost side of an arrow. We provide an
injection from positive to negative types: a value of type \code{'a} is also a
constant computation.

We do \emph{not} provide an injection from negative types to positive types:
this would allow nested arrows, that is, higher-order types.  One can take the
example of the \code{map} function, which has type \code{('a -> 'b) -> 'a list
-> 'b list}: we explicitly disallow representing the \code{'a -> 'b} part as a
\code{Fun} constructor, as it would require us to synthesize instances of a
function type. Rather, we ask the user to represent \code{'a -> 'b} as a
\code{Ty} constructor; in other words, we ask the user to supply their own test
functions as if they were a built-in type.

Since our GADT does not accurately model tagged, n-ary sums of OCaml, we provide
a last \code{Bij} case that allows the user to provide a two-way mapping between
a built-in type (say, \code{'a option}) and its \arti representation (\code{() +
'a}). That way, \arti can work with regular OCaml data types by converting them
back-and-forth.

This change of representation incurs some changes on our evaluation functions
as well. The \code{eval} function is split into several parts, which we detail
right below.
%
\begin{ocamlcode}
let rec apply: type a b. (a, b) neg -> a -> b list =
  fun ty v -> match ty with
  | Fun (p, n) ->
      produce p |> concat_map (fun a -> apply n (v a))
  ...
and produce: type a. a pos -> a list =
  fun ty -> match ty with
  | Ty ty -> Ty.elements ty
  | Prod (pa, pb) ->
      cartesian_product (produce pa) (produce pb)
  ...
let rec destruct: type a. a pos -> a -> unit =
  function
  | Ty ty -> (fun v ->
      remember v ty)
  | Prod (ta, tb) -> (fun (a, b) ->
      destruct ta a;
      destruct tb b)
  ...
\end{ocamlcode}
%
Let us first turn to the case of \emph{values}. In order to understand what \arti
ought to do, one may ask themselves what the user can do with values. The user
may destruct them: given a pair of type \code{'a * 'b}, the user may keep just
the first element, thus obtaining a new \code{'a}. The same goes for sums. We
thus provide a \code{destruct} function, which breaks down positives types by
pattern-matching, populating the descriptions of the various types it encounters
as it goes.

Keeping this in mind, we must realize that if we can use a positive type
\code{'a} to obtain a \code{'b} (\code{apply}), the user may use any possible
means to produce an \code{'a}: if \code{'a} is a product, they will use every
possible combination of elements that are available to them; if \code{'a} is a
sum, they will either every choice from the type of either. We must therefore
devise a function \code{produce} that represents the entire set of possible
choices for a positive type.

The \code{apply} function, just like before, takes a \emph{computation} along
with a matching description, and generates a set of \code{b}. However, it now
relies on \code{product} to exhibit all possible instances of a type before
passing these instances on to the actual function.

We are now able to accurately model a calculus rich enough to test realistic
signatures involving records, option types, and various ways to create
functions.

\subsection{Efficient representation of a set of instances}
\label{sec:efficient}

At this stage, the set of features \arti offers is relatively satisfactory;
severe performance problems remain, however. Take the case of a module that
exports logical formulas, for instance. With only three combinators and twenty
constants, a first pass would generate 1200 formulas with one combinators. Using
these as an initial set, a second pass would generate more than 4,000,000
formulas, all of which would contain at most two combinators. In order to get to
interesting formulas that contain all three combinators, a third round of
combinations would be required, at which point the OCaml runtime crashes.

The (assuredly naïve) scenario above highlights two points.
\begin{itemize}
  \item One may wish for a better strategy, where the new instances generated by
    the first combinator are fed to the second combinator, thus making sure that
    by the time we reach the third combinator, at least \emph{some} instances
    contain all three combinators. We will see that this approach is not
    feasible, because of non-termination and a lack of fairness.
  \item One may also argue that we need a better representation for our sets of
    instances. This is indeed the case. In this section, we propose a symbolic
    algebra for \emph{sets of instances} that \emph{mirrors} the structure of
    positive types.
\end{itemize}
%
We first turn to the latter point, then tackle the issue of \emph{instance
propagation}.
